 \documentclass[12pt]{article}

\usepackage{amssymb,amsmath,amsfonts,eurosym,geometry,ulem,graphicx,caption,color,setspace,sectsty,comment,footmisc,caption,natbib,pdflscape,subfigure,array,hyperref}
\usepackage{hyperref}
\hypersetup{colorlinks,
citecolor=blue,
linkcolor=magenta
}
\usepackage{multirow}
\usepackage{booktabs}
\usepackage{graphicx}
\usepackage{lscape}
\usepackage{setspace}
\usepackage{tabularx} 
\usepackage{rotating}
\usepackage{tablefootnote}
\usepackage{amssymb}
\usepackage{array}
\usepackage{ragged2e}
\usepackage{nicematrix}
\usepackage{enumitem}
\usepackage{array} 
\usepackage{graphicx}
\usepackage{booktabs}
\usepackage{enumitem}
\usepackage{wasysym}
\usepackage{framed}
\usepackage{pgfgantt}
\usepackage{subfloat}
\usepackage{multirow}
\usepackage{blindtext}
\usepackage{graphicx}
\usepackage{lscape}
\usepackage{colortbl}
\usepackage{caption}

\normalem

\doublespacing
\newtheorem{theorem}{Theorem}
\newtheorem{corollary}[theorem]{Corollary}
\newtheorem{proposition}{Proposition}
\newenvironment{proof}[1][Proof]{\noindent\textbf{#1.} }{\ \rule{0.5em}{0.5em}}

\newtheorem{hyp}{Hypothesis}
\newtheorem{subhyp}{Hypothesis}[hyp]
\renewcommand{\thesubhyp}{\thehyp\alph{subhyp}}



\geometry{left=1.0in,right=1.0in,top=1.0in,bottom=1.0in}
\bibliographystyle{aaai-named}

\singlespacing
\title{Has financialization changed the impact of macro announcements ?%\thanks{The authors thank participants at the Commodity \& Energy Markets Association 2021 meetings and World Finance \& Banking Association 2021 meetings as well as Alessandro Melone and Jocelyn Grira (discussants). For financial support, the authors thank the Social Sciences and Humanities Research Council and Chaire Industrielle-Alliance Groupe financier. Any remaining errors are ours alone.}}
%J'arrive pas le centrer dans le titre
%\author{Simon-Pierre Boucher\footnote{Corresponding author, PhD student in finance, Université Laval, Quebec City QC Canada G1V 0A6, email:   \texttt{simon-pierre.boucher.1@ulaval.ca}}\and Marie-H{\'e}l{\`e}ne Gagnon\footnote{Professor of Finance and Research Fellow, CRREP, Université Laval, email: \texttt{marie-helene.gagnon@fsa.ulaval.ca}}\and Gabriel J. Power\footnote{Professor of Finance and Research Fellow, CRREP and CRIB, Université Laval, email:   \texttt{gabriel.power@fsa.ulaval.ca}}. 
}
\author{BIND AUTHORS}
\date{ \today \\ First version: June 2021}
\begin{document}
\begin{titlepage}
\maketitle





\begin{abstract}
\noindent 
\singlespacing
This paper investigates the impacts in high- frequency of macroeconomic announcements on commodity futures returns and volatility while accounting for the financialization of commodities. We find financialization lessens the impact of macro news on commodity markets, as measured by price drift and volatility changes. This is consistent with prior literature suggesting that financial participants improve liquidity and price discovery and reduce volatility. Assuming traditional market participants prefer stability, our results suggest a beneficial impact of financialization, especially for pro-cyclical commodities.

\vspace{0.2in}
\noindent\textbf{Keywords:} commodities, futures,  spillover, financialization, high-frequency, commercial, institutional, volatility, macro, announcements, surprise, events\\


\bigskip
\end{abstract}
\setcounter{page}{0}
\thispagestyle{empty}
\end{titlepage}
\pagebreak \newpage




\doublespacing
\section{Introduction} \label{sec:introduction}
Since the early to mid 2000s, commodity derivatives have gained popularity among non-traditional market participants such as hedge funds and index traders. The financialization of commodities represents the phenomenon in which commodities are traded as financial assets by institutional and other non-traditional investors, either for diversification or speculative purposes. In general, the increase in speculation and indexation in the commodity market are pointed out as being responsible for an increase in price volatility and correlation between the different commodities, respectively. Useful surveys include \citet{boyd2018update} and \citet{cheng2014financialization}.

Since the financialization of commodities coincided with a historically significant commodity bull cycle, traditional market participants have voiced concerns of distortions away from fundamental values. \citet{masters2009testimony} was the first to argue that institutional investors have disrupted commodity markets by using investment strategies usually exclusive to the capital and money markets This \textbf{Masters hypothesis} has been however empirically contested \citep{irwin2011index,irwin2012testing,irwin2012financialization}. According to \citet{cheng2014financialization}, it is through risk sharing and information discoverythat financialization affects how commodity futures market function. In terms of risk sharing, financial investors can either provide liquidity to meet the hedging needs of other traders or consume liquidity when they trade for their own needs \citep{kang2020tale} and thus influence liquidity risk. On the information discovery side, the commodities market is affected by informational frictions in global supply, demand and stocks of commodities. In fact, globalized futures markets provide additional information to more decentralised spot markets. In consequence, financialization could alter how commodity markets incorporate new information in real time.

Knowing these two channels of transmission of macroeconomic announcements, we investigate the impact of financialization on commodity markets through the lens of macroeconomic announcement releases and highfrequency data. We assess the extent to which financialization  affects how commodity futures prices and volatility react to macroeconomic announcements. This issue concerns the  resolution of uncertainty, market efficiency, and how news is anticipated. Our paper builds on the financial literature on  information transmission in markets. \citet{goldstein2019commodity} show how financial participants in commodity markets affect future price informativeness, bias, and comovement. Their model predicts that financialization initially improves, but eventually worsens, price efficiency measured through volatility. This paper also relates to the evidence presented in \citet{brunetti2016speculators} pertaining to the positions held by certain types of speculators. They show that hedge funds benefit commodity markets by providing additional liquidity, resulting in a more efficient price. In contrast, merchant positions are positively correlated with volatility in the crude oil and natural gas market. 

The literature is mixed on the potential increase in volatility in the commodity market caused by financialization. \citet{singleton2014investor} concludes that information friction and prices moving away from economic fundamentals lead to a large increase in volatility. Using a Granger causality test and forecasting on the error of variance, \citet{yang2005futures} shows that an increase in commodity future trading volume leads to an increase in the volatility of commodity spot prices. \citet{bryant2006causality} reject the hypothesis that speculation and uninformed traders affect the level of volatility. The authors show that the two theories predicting the hypothesis (hedging pressure and normal backwardation theory) are rejected and have no explanatory power. Finally, using a conditional variance measure as well as daily returns on raw material futures, \citet{bohl2013does} show that financialization implies no change in variance.

Our contribution is at the intersection of two areas of expertise treated in the literature: macro announcements and financialization. Regarding macroeconomic announcements, high frequency data are widely used in the literature for bonds  \citep{andersen2007real, hu2013noise, balduzzi2001economic,lee1995oil, hautsch2011impact, kurov2019price}, stocks \citep{andersen2007real,bernile2016can,kurov2019price} and foreign exchange \citep{lee1995oil,andersen2003micro}. However, in the case of commodity futures, the use of high frequency data is not widespread. We argue that the use of a lower frequency for future returns in the financialization literature may possibly explain why previous research has not confirmed whether financialization increases or decreases volatility \citep{tang2012index,brunetti2016speculators,irwin2012testing,stoll2010commodity,alquist2013role}. Since most of of these studies use of daily and monthly data, the resulting sample is small, which reduces the power of statistical tests \citep{irwin2009devil}.To better measure the impact of financialization, we use the setting of high-frequency macro announcement releases, which is a useful way to look at information diffusion. The use of high-frequency data allows us to quantify permanent shock on volatility but also the non-permanent shock resulting from a new.

Moreover, the use of high-frequency data allows us to bypass the traditional criticism of event-study methods, which claim that an announcement having an impact on the daily return of a financial asset can easily be attributable to another phenomenon \citep{kothari2007econometrics}. We focus on the USA due to the greater availability of high-frequency data on commodity futures as well as the wealth of distinct macro announcements.

Our first contribution is based on our measures quantifying the level of financialization. Most previous studies have assumed that financialization started in the early 2000s and that separating the sample into two groups (pre-financialization and post-financialization) is sufficient to capture the impact of this phenomenon \citep{buyukcsahin2010matters, kilian2014role,brunetti2016speculators,irwin2012financialization,stoll2010commodity,alquist2013role}
. While it is true that the era of financialization seems to have started in the 2000s, the use of this method seems to neglect the time varying speculation level in the post-financialization period. We further document these measures by separately quantifying the level of speculation of fund managers and swap dealers.

 Our findings are as follows: We document that financialization contributes to information diffusion and price discovery. we find that in general, the effect of macro announcements is attenuated when financialization has a greater presence for a given commodity. We interpret this result as indicating that commodity markets are better informed–and macro news create less of a shock–when they are financialized, which should be beneficial for traditional commodity market participants. This main finding is weaker in the case of gold, a safe haven asset. Thus, the beneficial effects of financialization appear to be stronger for pro-cyclical commodities.

Second, we find that money managers seem to contribute to price discovery when there is a macro announcement, while helping to reduce volatility. This result is consistent with the fact that money managers are more informed investors given their function in the markets. Swap dealers also contribute to the price discovery but  cause an increase in volatility following macroeconomic announcements. When we consider all categories of traders, greater levels of financialization seems to reduce  volatility in the commodity markets. 

In terms of policy implications, our results reject the \citet{masters2009testimony} hypothesis stating that financialization increases commodities price instatbility and systemic risk, consistent with a considerable share of the literature. However, when we separate financial traders into two categories, our results allow us to reject only the \citet{masters2009testimony} hypothesis for money managers. The other findings suggest that the \citet{masters2009testimony} hypothesis may be confirmed for swap dealers. Robustness tests are performed demonstrating our results also robust when we use an alternative volatility measures and differences in estimation specification.

\section{Literature Review}

\subsection{Financialization of Commodities}

Several contributions in the literature argues that the financialization of commodities has a clear impact on futures prices and volatility. \citet{tang2012index} show that the popularity of index funds and the rapid growth of index investments in the commodity market have increased the correlation between oil prices and the prices of other non-energy commodities. They argue that when a commodity is included in a benchmark index, its individual price is no longer determined solely by the principle of supply and demand governing that specific commodity, but also by all the commodities and other financial assets included in the index. Using an equilibrium model and a dataset containing traders’ positions in the commodity futures market, \citet{brunetti2014commodity} show that index traders represent an important source of insurance against price risks.\citet{basak2016model} develop a model and find that: (i) the prices of all commodity futures go up with financialization and that this increase is more significant for futures belonging to the index than for nonindex futures. (ii) The volatility of both index and nonindex futures return go up with financialization. (iii) correlations among commodity futures as well as the equity-commodity correlations increase with financialization. (iv) Only storable commodity prices are affected by financialization. Moreover, in the presence of institutional investors, storable commodity inventories and prices are higher than in the benchmark economy, and again this effect is stronger for commodities included in the index.


In terms of speculation, Singleton (2014) shows that speculative activity by financial investors creates significant informational frictions such that commodity prices can quickly move away from the value justified by fundamentals. 

However, a concensus has not been reached in the literature.  \citet{stoll2010commodity}  show that inflows and outflows from commodity index investments do not cause significant changes in price and volatility in the Granger sense.. Using an arbitrage argument, \citet{hamilton2014risk} show that the positions of commodity traders included in index funds cannot be used in order to achieve excess return in the futures market. On the side of financialization linked to speculation,  \citet{kilian2014role}  show that several speculative trades can occur in the oil market without observing a significant change in inventory levels, ruling out speculation as responsible for the boom and bust in the oil market between 2003 and 2008. Subsequently,  \citet{brunetti2016speculators}  analyze the impact of certain types of speculators on the commodity market from 2005 to 2009. Their findings are that hedge funds allow for faster and efficient price discovery, which results in reduced volatility of returns. Furthermore, they find that the positions of swap dealers are not correlated with contemporaneous returns and volatility in the commodity markets.  Table 1 presents a summary of the literature on the influence of financialization and speculation on the volatility of commodity futures returns

\subsection{Macro News Announcements}

Other research has examined whether economic conditions can explain some of the more heterogeneous responses across business cycles \citep{boyd2005stock,andersen2003micro}. \citet{boyd2005stock} shows that the impact of announcements related to unemployment has a different impact depending on the economic situation. \citet{andersen2003micro} also find that the stock market reacts to news differently depending on the stage of the business cycle.

Exploiting the fact that high-frequency traders receive the Michigan Consumer Sentiment Index 2 seconds before its official announcement, \citet{hu2017early} show that there is evidence of highly concentrated trading and rapid price discovery occurring within 200 milliseconds. Outside this narrow window, typical investors trade at fully adjusted prices. 

\citet{balduzzi2001economic}  use intraday data from the interdealer government bond market to investigate the effects of scheduled macroeconomic announcements on prices, trading volume, and bid-ask spreads. They find that 17 public news releases, as measured by the surprise in the announced quantity, have a significant impact on the price of at least one of the following instruments: a three-month bill, a two-year note, a 10-year note, and a 30-year bond. In addition, they argue that these effects vary significantly according to maturity. The authors also show that following a macroeconomic announcement, volatility and trade volume increase persistently and significantly.


While the literature documenting impacts of macroeconomic announcements on stock \citep{boyd2005stock,andersen2003micro,hu2017early,scholtus2014speed} and bonds \citep{fleming1997moves,fleming1999price,balduzzi2001economic} is quite large (see e.g. cite quelques papiers que j’ai enlevé), the commodity literature has mostly focused on OPEC meeting announcements so far.  \citet{horan2004implied} find that volatility drifts upward as the OPEC meeting draw nearer, then decreases by three percent after the first day of the meeting and by five percent over a five-day window period. \citet{wirl2004impact} investigate how far OPEC might influence world oil markets by looking at fifty meetings from 1984 to 2001. They find that the market does not seem to reveal much about these meetings. \citet{kilian2011energy} regress the WTI crude oil and U.S. gasoline prices on 30 U.S. macroeconomic announcements using daily data from 1983 to 2008 and find no evidence of statistically significant responses of either oil or gasoline to U.S. macroeconomic news at daily time horizons.


At the level of macroeconomic announcements, \citet{frankel1985commodity} find using daily data on commodities that 1 percentage point positive shock in the money supply leads to a 0.7 percent decline in gold. \citet{christie2000macroeconomics} analyzes sensitivity of gold and silver futures prices over 15-minute intervals to 23 U.S. macroeconomic news announcements from 1992 to 1995 and find that gold and silver price volatility is higher during days in which there are announcements. \citet{cai2001moves} capture intraday patterns of 5 minutes gold price using GARCH models aroundmacroeconomics announcements. They find GARCH effects in intraday prices, but the number and significance of announcements coefficients are lower for gold than for bonds or currencies. In connection with financialization, \citet{hess2008commodity} use an OLS regression to quantify the impact of 17 US macroeconomic announcements on the price of commodity indices (CRB and Goldman Sachs index). They find that commodity indexes are sensitive to far fewer announcements than bonds or stocks.

\citet{gu2018drives} show that, in the natural gas market, the difference between the median forecast of analysts with high historical forecasting accuracy and the consensus forecast can be used to predict inventory surprises, which explains some of the pre-announcement price drift.  \citet{hollstein2020volatility} analyze the impact of selected economic variables on the term structure’s volatility in commodity future markets. They show that speculation and macroeconomic variables related to employment have the largest impact on volatility. Lastly, \citet{ye2021macroeconomic} quantify the impact of expectations about future economic conditions on commodity futures markets. They find that volatility in commodity futures seems to be more impacted by macroeconomic forecasts than by current economic conditions. 

\section{Data}
\subsection{Commodity Price Data}
For each of the commodities studied, the return is calculated over a period of 5 minutes $(\tau=5)$ using intraday data. We designate Rt as the return over a period of 5 minutes starting exactly at time t. In the database we have, for all 5-minute intervals, the price of the futures contract at the opening ($p_{t}^{Open}$) of that 5-minute period and at the close ($p_{t+\tau}^{Close}$). Subsequently, the return $R_t$ is obtained by using $p_{t}^{Open}$ and $p_{t+\tau}^{Close}$ as in equation (\ref{eqn:RETURN}):

\begin{equation}\label{eqn:RETURN}
R_t^{t+\tau}=\ln \left( \frac{p_{t+\tau}^{Close}}{p_{t}^{Open}} \right)=\ln (p_{t+\tau}^{Close})-\ln(p_{t}^{Open})
\end{equation}

Our last analysis seeks to confirm that the results obtained using 5-minute returns can also be obtained if we use returns calculated over a larger time interval. To do this, we estimate the equation 10 again, but this time using 30-minute returns.

Typically, Crude Oil and Natural Gas (energy sector) have a pro-cyclical behavior. On the other hand, Gold and Silver (precious metals sector) behave as a safe haven while Copper and Palladium (industrial metals sector) behave ambiguously as they are used in the manufacturing of consumer products.

\subsection{Macroeconomic Announcements}
 
The 22  announcements that we use are standard to the literature (see e.g.,  \citep{kurov2019price}). We can separate the announcements into 10 main categories: Income, Employment, Industrial Activity, Investment, Consumption, Housing Sector, Government, Net Exports, Inflation and Forward-looking. The majority of announcements are released on a monthly basis. However, there are some exceptions, for which the frequency of release is quarterly or weekly.

Following the literature on macro announcements, we do not directly use the value of the announcement release, but rather the resulting \textbf{surprise}. For the calculation of announcement surprises, we use the method in \citet{balduzzi2001economic} as a starting point. The variable $A_{kt}$ represents the realized value for macroeconomic announcement k at time t, while $E_{kt}$ represents instead the median value of all analysts’ forecast for the macroeconomic announcement $k$ at time $t$. In addition, $\sigma_k$ represents the sample standard deviation of the surprise in absolute value, for the entire period of our sample and the macroeconomic announcement $k$. Equation (\ref{eqn:SURPRISE}) presents the surprise at time $t$ for the macroeconomic announcement $k$.

\begin{equation}\label{eqn:SURPRISE}
S_{kt}=\frac{A_{kt}-E_{kt}}{\sigma_k}
\end{equation}

Table \ref{tab:stat1} summarizes each announcement along with its category, frequency, source, unit of measure, and release time respectively. Bloomberg provides analyst forecasts for all macroeconomic announcements. For the actual value of the macroeconomic announcement release, data are available from Bloomberg (see e.g., \citep{kurov2019price}). 

In Table \ref{tab:stat2}, we present the minimum value, 1st quartile, median, mean, 3rd quartile and maximum value of the surprise for each macroeconomic announcement. Note that the considerable difference between the minimum value and the 1st quartile, as well as the difference between the maximum value and the 3rd quartile, explained in large part by macro announcements that took place during the COVID-19 pandemic. Figures 5, 6, 7 and 8 present, for each macroeconomic announcement, a time series displaying the median analyst forecast and the realized value. Note that analyst forecasts are relatively precise, even during a period of high volatility in realized value.

\subsection{Measures of commodity financialization}
In order to measure the impact of commodity financialization, we need a measure of the intensity of speculation in commodity markets. We argue that commodity markets are more financialized when speculative activities increase relative to production activity. The indexes are constructed using data in the \textbf{Commitment of Traders (COT) Report} published weekly by the Commodity Futures Trading Commission (CFTC). The Commodity Futures Trading Commission (CFTC) provides a comprehensive database with weekly observations. The data provided by the CFTC is only related to the number of positions held by different types of participants in commodity markets. The CFTC separates the types of traders as follows:
\begin{enumerate}
\item \textbf{Commercial}: All trader’s reported futures positions in a commodity are classified as commercial if the trader uses futures contracts in that commodity for hedging
\item \textbf{Non-Commercial:} Derived by subtracting total long and short Commercial Positions from the total open interest
\end{enumerate}
The Commodity Futures Trading Commission defines commercial traders as participants in commodity markets who primarily use futures contracts to hedge their business activities (e.g., buying or selling commodities). All traders who are not classified as Commercial are automatically be classified as Non-Commercial traders. To obtain the number of long positions held by non-commercial traders, we subtract the total long Commercial Positions from the total open interest. For the number of short positions held by non-commercial traders, we subtract the total short Commercial Positions from the total open interest. The following elements are presented in the report and are used as inputs for financialization measures:
\begin{itemize}
\item $SS_i$ is the number of short positions in commodity i futures held by non-commercial traders,
\item $SL_i$  is the number of long positions in commodity i futures held by non-commercial traders,
\item $HS_i$ is the number of short positions in commodity i futures held by commercial traders, 
\item $HL_i$ is the number of long positions in commodity i futures held by commercial traders.
\end{itemize}
\subsubsection{Working-T}
We use a measure developed by \citet{working1960speculation}, which allows for the comparison between speculative activities and hedging activity. More specifically, the index compares the level of activity of non-commercial commodity future traders (e.g., speculators) to the activity of commercial traders (e.g., hedgers). Typically, commercial traders take short positions in futures contracts while non-commercial traders take long positions (see also \citet{shanker2017new}, for an updated definition of Working’s T).. Working’s model measures the extent to which speculation exceeds the level required to offset any unbalanced hedging at the market clearing price. 
The Working-T index, $WT_i$, can be constructed using equation (\ref{eqn:Working}) :

\begin{equation} \label{eqn:Working}
WT_i\left\{\begin{matrix}
 1+\frac{SS_i}{HL_i+HS_i} \hspace{0.5cm} \mbox{if} \hspace{0.5cm} HS_i \ge HL_i\\
1+\frac{SL_i}{HL_i+HS_i} \hspace{0.5cm} \mbox{if} \hspace{0.5cm} HS_i < HL_i
\end{matrix}\right.
\end{equation}


\subsubsection{Market Share of Non-Commercial (MSCT)}
\citet{buyukcsahin2014speculators} propose a measure of commodity financialization showing the market share of non-commercial traders. The market share of non-commercial traders (MSCT) is a ratio between the sum of the short and long positions of non-commercial traders over twice the total open interest in that market: 
\begin{equation} \label{eqn:MSCT}
MSCT_i=\frac{SL_i+SS_i}{2 \times OI_i}
\end{equation}

\subsubsection{Net Long Short (NLS)}
\citet{hedegaard2011margins} uses an index measuring the extent of speculation by calculating the ratio of net long speculative positions over total open interest ($NLS_i$)
\begin{equation} \label{eqn:NLS}
NLS_i=\frac{SL_i-SS_i}{OI_i}
\end{equation}

Tables \ref{tab:stat5}, \ref{tab:stat6} and \ref{tab:stat7} present the descriptive statistics for the financialization variables NLS, MSCT and WORKING-T respectively. We can see that palladium has the highest average value for our three financialization variables. Furthermore, when we look at the minimum and maximum value, we also see that palladium has the lowest minimum value and the highest maximum value. The level of financialization of palladium seems to be very volatile compared to other commodities. In Figures \ref{fig:MSCT}, \ref{fig:NLS} and \ref{fig:WORKINGT}, for each commodity, a time series shows respectively the MSCT index, the NLS index and the Working-T index. Crude oil is the only commodity for which the three indices, measuring the degree of financialization, are increasing over time. This observationis consisten suggests that crude oil has attracted more financial interest from non-traditional market participants than other commodities. Returns autocorrelation is only negative for natural gas, and only occasionally. In addition, all commodities in our sample have a seasonality component in the autocorrelation of returns.
 
Next, Tables \ref{tab:stat8}, \ref{tab:stat9} and \ref{tab:stat10} present the correlation between the different commodities for the financialization variables NLS, MSCT and WORKING-T respectively. There is a strong correlation of the financialization variable MSCT and WORKING-T between Silver, Gold and Copper. Indeed, for the MSCT variable, the correlation is 15.02\% between silver and gold and 14.53\% between silver and copper. When we look at the WORKING-T variable, the correlation between silver and gold is 32.60\% and the correlation between silver and copper is 9.73\%. For the financialization variable NLS, there does not seem to be a significant correlation between the different commodities.

\section{Methodology}
\subsection{Modeling the impact on returns}\label{return}
Our regression models are based on \citet{kurov2019price} and \citet{andersen2007real}. To be consistent with the existing literature, we perform the two procedures for the regression model presented in the equation (\ref{eq:Model 1}):
\begin{equation}\label{eq:Model 1}
R_{t}^{t+\tau}=\alpha+\sum_{m=1}^{22} \gamma_m S_{m,t}+ \delta X_{t,i} + \sum_{m=1}^{22} \theta (S_{m,t} \times X_t)+\beta R_{t,-\tau}^{t}+\epsilon_{t} 
\end{equation}
In equation (\ref{eq:Model 1}), $R_{t}^{t+\tau}$ denotes the continuously compounded asset return between time $t$ and $t+\tau$, $S_{mt}$ denotes the surprise for the macroeconomic announcement $m$, which was published at time $t$. $X_{t,i}$ is the measure of commodity financialization $i$,  which is valid at time $t$. $X_{i = 1}$, The three possible values for the exogenous variable are:  $X_{t,1}$, $X_{t,2}$ and $X_{t,3}$ represent the financialization variables $MSCT_i$, $NLS_i$ and $WT_i$ . 

We estimate the model using the two-step weighted least squares procedure. In the case of \citet{andersen2007real}, we estimate the equation (\ref{eq:Model 1}) by OLS. Then, we regress the residuals of the last model, in absolute value, on the macro variables as well as on 23 times dichotomous variables to adjust for the time of day. This auxiliary regression is represented by the equation (\ref{eq:auxiliary 1}).

\begin{equation}\label{eq:auxiliary 1}
\mid \epsilon_{t} \mid=\rho+\sum_{m=1}^{22} \zeta_m S_{m,t}+\sum_{h=1}^{23} \delta_h D^h
\end{equation}
After estimating the model, we use the fitted value of the residuals to obtain the WLS regression weight:
\begin{align*}
w_t=1/\mid \hat{\epsilon_t} \mid^2
\end{align*}

We finish by multiplying each dependent and independent variable of our original model by $w_t$, to finally estimate the model again by OLS.

Next, we estimate eq.(\ref{eq:Model 1}) using\citet{kurov2019price} approach. heteroscedasticity is accounted for by constructing an estimate for the volatility via an exponential moving average, using the regression residuals obtained in the first step. This auxiliary regression is represented by the equation (\ref{eq:auxiliary 2}) in which the smoothing parameter is $\alpha=0.9$ and the starting parameter is set as $\sigma_1=\epsilon_t$:
\begin{equation}\label{eq:auxiliary 2}
\sigma_t=\alpha \sigma_{t-1}+(1-\alpha) \mid \epsilon_t \mid 
\end{equation} 
After finding $\sigma_t$ for each observation, we transform it to obtain the WLS regression weight.
\begin{align*}
w_t=1/\mid \hat{\sigma_t} \mid^2
\end{align*}
We finish by multiplying each dependent and independent variable of our original model by $w_t$, to finally estimate the model again by OLS.

 The impact of macroeconomic announcements on the commodity futures returns  can be assessed by looking at the significance of our coefficient $\gamma_m$ in the mean equation. As for the impact of the financialization of commodities on the commodity futures return, we test the significance of the coefficient $\delta$. Finally, the significance of the coefficient $\theta_m$ is tested to quantify the simultaneous impact of the financialization of commodities as well as of the surprise following a macroeconomics announcement on the commodity futures return.
 
\subsection{Modeling the impact on volatility}\label{variance}
Using a GARCH specification is justified by the time-varying and clustered volatility of commodity price returns (e.g. \citep{hammoudeh2008metal}).

The ARCH(1) specification is used to quantify the impact of macro announcements and financialization on the conditional variance. The estimation of the ARCH model is done in two steps: First, we estimate the mean eq. (\ref{eq:Mean}):
\begin{equation}\label{eq:Mean}R_{t}^{t+\tau}=\alpha+\sum_{m=1}^{22} \gamma_m S_{m,t}+\beta R_{t,-\tau}^{t}+\epsilon_{t}
\end{equation}


\begin{equation}\label{eq:Variance}
h_{t}=\alpha_0+\alpha_1 h_{t-1}+\sum_{m=1}^m \Phi_m D_{m,t}+\beta X_{i,t}+\sum_{k=1}^n \phi_k I_{kt}
\end{equation}

where $I_{kt}=D_{m,t} \times X_{i,t}$ and $D_{m,t}$ represents a dummy variable for the macroeconomic announcement $m$, taking the value $1$ if an announcement took place at time t and 0 otherwise. $X_{i,t}$  represents the financialization variable $i$ valid at time $t$. The impact of the macroeconomic announcements m on the commodity futures volatility is tested through the significance of $\Phi_m$ in the variance eq.(\ref{eq:Variance}). To assess the impact of the financialization variable $X_m{i,t}$ on commodity futures volatility, we look at the significance of the coefficient $\beta$. Finally, we look at the significance of the coefficient $\phi_k$ to assess the simultaneous impact on commodity futures volatility of the financialization of commodities and the surprise contained in macroeconomic announcement $m$. 

Among the announcements that are analyzed, only a positive surprise for Initial Jobless Claims indicates a deterioration in economic conditions. In the case of the energy sector, we expect the surprise coefficient to be positive if the surprise for the Initial Jobless Claims announcement is negative. For the other announcements, the coefficient is expected to be positive when the surprise is positive. For precious metals, the coefficient attached to the surprise is expected to be positive if the surprise for the \textbf{Initial Jobless Claims} announcement is positive. For the other announcements, the coefficient is expected to be positive when the surprise is negative.

\subsection{Types of non-commercial traders}
In order to better categorize financialization and its effects, we reproduce the procedure in (\ref{return})  and (\ref{variance}) using only the NLS index for two separate groups of Non-Commercial investors: swap dealer and money managers. A swap dealer is an entity that deals primarily in swaps for a commodity and uses the futures markets to manage or hedge the risk associated with those swaps transactions. The swap dealer’s counterparties may be speculative traders, like hedge funds, or traditional commercial clients that are managing risk arising from their dealings in the physical commodity. A money manager is a registered commodity trading advisor (CTA), a registered commodity pool operator (CPO), or an unregistered fund identified by CFTC. These traders are engaged in managing and conducting organized futures trading on behalf of clients. For both categories (swap dealers and managed money), the CFTC reports the number of long and short positions. We construct, for each category, the NLS index proposed by Hedegaard (2011). This measurement will allow us to quantify the extent of speculation for Money Manager and Swap Traders respectively. For Swap Traders, we will represent the index by $NLS_{SWAP}$ while for Money Mangers, we will represent the index by $NLS_{MM}$.

\section{Empirical results} \label{sec:result}

\subsection{Effect on returns}
We now present the results of regression explaining returns following a macroeconomic announcement. Results for different commodities are presented in tables \ref{tab:reg1} (crude oil), \ref{tab:reg2} (gold), \ref{tab:reg3} (gold), \ref{tab:reg4} (copper), \ref{tab:reg5} (natural gas) and \ref{tab:reg6} (silver), respectively.

We focus first on the $\gamma_m$ coefficients which measures the impact of a surprise on the return of commodity futures following a macroeconomic announcement. The coefficient attached to the surprise for the Initial Jobless Claims announcement is negative in the case of Crude Oil (table 12) and positive in the case of Gold (table 13). This is consistent with oil being procyclical while gold is typically seen as a safe haven asset.  In addition, this result is significant for the coefficients obtained using the method of \citet{kurov2019price} and \citet{andersen2007real}. For the surprises linked to the CB Consumer, Advance Retail Sales, ADP Employment and Pending Home Sales announcements, we obtain coefficients that are significant and positive for Crude Oil (table \ref{tab:reg1}) and significant and negative for Gold (table \ref{tab:reg2}). Table (\ref{tab:reg3})  shows that Copper returns behaves like Crude Oil returns, as high-grade copper is an industrial metal and a pro-cyclical commodity. The coefficient attached to a surprise of the Initial jobless claims announcement is negative but positive for other macroeconomic announcements. 

Next, we consider each of the three candidate variables to proxy for financialization, one of which is added to the regression model eq. (\ref{eq:Model 1}) . 

The $\theta_m$ coefficient evaluates the impact of financialization on commodity returns following an announcement release. For Crude Oil (table \ref{tab:reg1}) and Gold (table \ref{tab:reg2}), while $\gamma_m$ is similar to the previouswithout the financialization variable,the coefficient $\theta_m$ is systematically of the the opposite sign.Thus, the increase in speculation reduces the extent of the drift during a macroeconomic announcement. This effect is robust to the other financialization variables NLS and Working’s-T as presented in tables \ref{tab:reg1} through \ref{tab:reg6}.

\subsubsection{Financialization effects for particular macro announcements}

 
  In this subsection, we present the macroeconomics announcements that seem to have an impact on all commodities futures returns, when we combine the impact of macroeconomics surprise with our financializations variables.  ADP employment  all the concideredcommodities for all financiarization variable considered. More specifically, for the financialization variable MSCT, the effect of financialization is significant, for all commodities, when we combine it with the surprises of the following macroeconomic announcements: ADP Employment, Durable goods orders and non-far employments. Then, for the NLS financialization variable, the effect of financialization is significant, for the following macroeconomic announcements: Initial jobless claims, ADP Employment, Advance retail sales, new home sales and Personal income. Finally, for the WORKING-T financialization variable, the effect of financialization is significant, for the following macroeconomic announcements: Initial job- less claims, ADP Employment, CB Consumer, Durable goods orders, new home sales and Non-farm employment. We can quickly notice that the announcements being related to employment and household income seem to have an impact on the returns of all commodities when we include one of the three financialization variables in the regression model. Our results are consistent with \citet{hordahl2015expectations} showing that the macroeconomic announcements included in the Employment Report being the most important and the most likely to influence the return and the volatility of the price of financial assets.
  
	\subsection{Effect on volatility}
	Table \ref{tab:var1} shows the results of the equation 10 estimated for Crude Oil futures. The $\Phi_m$ coefficient associated to announcement is positive and significant.Macroeconomic announcements typically increases crude oil futures volatility. However, $\phi_k$ coefficient  is always negative when it is significant supports \citet{brunetti2016speculators}, who argue that speculation reduces volatility instead of increasing it. Table \ref{tab:var2} shows a similar result for Gold. Indeed, the $\phi_k$  coefficient is not systematically negative when it is significant. For Copper, table \ref{tab:var3} shows a similar effect and increased speculation in copper lowers volatility considerably during a macro announcement. The results shown in tables \ref{tab:var4},\ref{tab:var5} and \ref{tab:var6} indicate that for Natural Gas, Palladium and Silver, the results seem to be of the same magnitude as those of Crude Oil.  Overall, our results are consistent with those obtained by \citet{brunetti2016speculators}.
	
	
	\subsubsection{Financialization effects for particular macro announcements}
At the level of conditional variance, when we combine the macroeconomic announcements Non-farm employment and Pending home sales with the MSCT financialization index, we obtain a consistent result for all the commodities studied. When we use NLS as an index of financialization, the surprise coefficient for Advance retail sales, Construction spending, Factory orders and Non-farm employment macroeconomic announcements, we obtain significant and consistent results for all commodities as well.
Finally, for the Working-T financialization index, Factory orders, Initial jobless claims and New home sales are the only macroeconomic announcements producing significant results for all commodities.

 
 %\vspace{1cm}
 
Price discovery following a surprise for the macroeconomic announcements contained in the Employment Report seems to be more efficient when the market is more financialized and this for all the commodity futures contracts studied. However, the results showing that the financialization favors a reduction in variance, are less generalizable to all the commodities studied. The only macroeconomic announcement with a negative and significant coefficient for all commodities is Non-farm employment. Once again, this results is consistent with those obtained by \citet{hordahl2015expectations}.

\subsection{Types of non-commercial traders}
 Tables
The results of our robust regressions involving equation \ref{eq:Model 1} (i.e., with financialization variables NLS : Money Manager and Swap Dealer) are presented in tables  \ref{tab:robut1} (crude oil), \ref{tab:robust2} (gold), \ref{tab:robust3} (copper), \ref{tab:robust4} (natural gas), \ref{tab:robust5} (palladium) and \ref{tab:robust6} (silver), respectively.  
 Subsequently, for the results concerning  variance, tables \ref{tab:robut1.b}, \ref{tab:robust2.b}, \ref{tab:robust3.b}, \ref{tab:robust4.b}, \ref{tab:robust5.b} and \ref{tab:robust6.b} present the results of our robust regressions involving equation \ref{eq:Variance} (i.e., with financialization variables NLS : Money Manager and Swap Dealer), for Crude Oil, Gold, Copper, Natural Gas, Palladium and Silver, respectively.
%Regarding the results of the robust regressions
 %\vspace{1cm}
 
These additional results confirm those we have already obtained with the initial methodology. In addition, by dividing financial traders into two groups, Money manager and Swap dealer, we also provide additional support for the results in \citet{brunetti2016speculators}. Indeed, it seems that the phenomenon whereby financial traders reduce volatility by limiting hedging pressure, is solely attributable to Money Managers. In contrast, our results show that swap dealers seem to worsen hedging pressure, which most often results in increased volatility.

 %\vspace{1cm}
 
For Crude Oil, one can see the phenomenon that we have just explained in table \ref{tab:robut1}. Concerning money managers, note that the coefficient $\gamma_m$ is positive when it is significant while the coefficient $\theta_m$ is negative when it is significant. These results once again confirm the procyclical nature of crude oil with a positive $\gamma_m$ coefficient for all announcements except Initial Jobless Claims. The fact that the $\gamma_m$ coefficient has the opposite sign to the $\theta_m$ coefficient suggests that money mamager contributes to reducing hedging pressure. Concerning the dealer swap,  note that the $\gamma_m$ and $\theta_m$ coefficients now have the identical sign.  Unlike the money manager, swap dealers seem to worsen hedging pressure and do not necessarily contribute as much to improving liquidity on futures contracts for crude oil.  Finally in table \ref{tab:robut1.b}, concerning money managers and swap dealers, you can notice that in column (a) that we get a negative coefficient when it is significant, while for column (b) we get a positive coefficient when it is significant. These results tell us that in addition to reducing hedging pressure during a macroeconomic announcement, money managers also seem to contribute to a reduction in volatility.  In contrast, swap dealers seem to worsen hedging pressure during a macroeconomic announcement and increase volatility as well. 
%\vspace{1cm}

The robust results we have just presented for Crude Oil are also valid for all commodities except gold. The main reason that can explain the less significant and sometimes contradictory results is based on the safe heaven attribute of gold.  Gold is described as a multifaceted asset due to its many attributes of currency,  commodity, and risk aversion \citep{wu2019does}. Studies have focused more on the last attribute as gold acts as safe haven in period of economic uncertainties and turbulent markets environment.  More specifically, \citet{baur2010gold} formulated the empirical observations that we should obtain for an asset class in order to consider it as a safe haven. To be considered as a safe haven, this asset must have returns that are uncorrelated or negatively correlated with another asset returns. This last property must also be valid only in times of market stress or turmoil. 

%\vspace{1cm}

Now knowing this properties of gold, financial traders will be especially interested in taking a long position in a gold futures contract in times of uncertainty or crisis. This last statement is confirmed by figure 3 (a) where we can see that the number of long positions in gold futures contracts is constantly increasing over time, while the number of short positions is substantially constant. and in very low proportion compared to other commodities. In contrast, non-financial traders have a relatively symmetrical distribution between long and short positions. Knowing the large proportion of long positions held at all times in future gold contracts by financial traders, the following two phenomena are possible:
\begin{itemize}
\item When non-financial traders are mostly long in the gold futures contract market: Financial traders will worsen the hedging pressure situation and thereby cause increased volatility.
\item When non-financial traders are predominantly in short positions in the gold futures market: Financial traders will be in a situation where their positions will be opposed to those of non-financial traders, which will likely result in a decrease in hedging pressure and therefore a decrease in volatility.
\end{itemize}


\subsection{Discussion of the results}
The results we obtained are not only consistent with the evidence presented inthose of \citet{goldstein2019commodity}, but they also provide an more detailed explanation. for the phenomenon documented in their research, \citet{goldstein2019commodity} use as a starting point the theories proposed by Grossman and Stiglitz (AER, 1980), Kyle (Econometrica, 1985), Glosten and Milgrom (JFE, 1985), to modelbetter understand how information can be incorporated into the price of financial assets. Theyir model uses the concept of asymmetric information where financial traders, commodity producers, and noise traders trade futures contracts. Their first results show that financial traders bring in new information when they enter the commodity futures market and. However, their results also show that the presence of financial traders can improve price accuracy measured in terms of precision, a function of the price variance. The lower the variance, the higher the precision and precision represented is simply the inverse of the variance. However, in some circumstances, they worsen  but also make the situation worse . Their conclusion is that financial traders also bring noise with the new information. The improvement in price accuracy caused by the new information brought by financial traders will dominate the loss of accuracy caused by noise when the proportion of financial traders remains relatively small compared to commercial traders. Thus, \citet{goldstein2019commodity} find results show that an increase in the proportion of financial traders up to a threshold point (around 20\%) will increase the accuracy of commodity future contract prices equivalent to a reduction in volatility. When the proportion of financial traders passes the threshold point, the noise brought in by financial traders becomes more important and dominant than the new information brought in.


In order to compare our results with these, it is important to specify that the precision of the prices can be expressed as a function of the variance of this same price. The more the values of the price of an asset are scattered around the average (high variance), the less precise they are (low precision).  The lower the variance, the higher the precision. The precision represented by $\tau$ is simply the inverse of the variance: $\tau=\frac{1}{\tau}$ 



In their theoretical model, they argue that financial traders are all the same and that the loss of accuracy past the threshold point is simply caused by too large a proportion of financial traders’ positions compared to commercial traders’ positions. 

However, our results provides further depth to their model by concidering in order to fully understand the phenomenon, it is important to consider the different types of traders with included in the financial trader category as they do not have the same level of risk aversion, the same objective and especially differentthe same regulatory restrictions. Our results show that if financial traders were composed solely of money managers, an increase in the proportion of financial traders past the threshold point would continue to improve price accuracy and thereby reduce volatility. On the other hand, if the financial traders were composed only of swap dealers, we would have a less accurate and more volatile price, whether the level of financial traders passed the threshold point or not. Overall, Oour results seem to indicates that the loss of accuracy or the increase in volatility is not necessarily due to a too high concentration of financial traders, but rather to the type of trader concidereds include in the financial trader’s category.

The fact that the precision of the price past the threshold point does not decrease to the level before the threshold point is consistent with our result showing that globally financialization reduces volatility following a macroeconomic announcement . In other words , price accuracy will still be better past the threshold point compared to a situation where we have no financial traders and just commercial traders.

Like \citet{brunetti2009speculation} this first result depends on a proxy for financialization that includes all financial investors. It is still possible for a specific class of trader to implement trading strategies that move prices and increased volatility. Knowing this, our results imply that financialization as a whole reduces volatility when there is a macro announcement. We interpret this result as indicating that commodity markets are better informed–and macro news create less of a shock–when they are financialized, which should be beneficial for traditional commodity market participants. Subsequently, we ex- amine the impact of different types of traders by breaking down the data. We find that money managers seem to contribute to price discovery when there is a macro announcement, while helping to reduce volatility. This result is consistent with the fact that money managers are more informed investors given their function in the markets. On the other hand, swap dealers also contribute to price discovery while causing an increase in volatility following macroeconomic announcements. Our second result is consistent with \citet{cheng2012convective}, who show that fund managers are clearly more sensitive to market information and fill hedgers’ liquidity needs by taking the opposite position. This result is also consistent with the results of \citet{goldstein2014speculation} showing that financial speculators improve price informativeness, while hedgers decrease it. Finally, all the results obtained are robust to the use of a non-parametric variance estimator. 
 



\section{Conclusion} \label{sec:conclusion}
 This paper investigated whether financialization has amplified the impact of macro announcements on prices or volatility in commodity markets. our results suggest that financialization is beneficial to commodity markets, by reducing volatility and improving price discovery. financializationdoes not appear to have amplified macro announcement releases effects. In fact, volatility fluctuations are mitigated after macro releases when there is greater financialization. Our results are consistent with literature suggesting that non-traditional investors such as hedge funds arebeneficial to commodity markets by supplying liquidity, reducing volatility, and improving market efficiency. Our results are robust to the use of a non-parametric variance estimator and to alternative empirical specification.
\newpage
\bibliography{master}
\section{Tables}
\section{Tables}
\begin{landscape}
\begin{table}[]
\caption{Summary of the literature: Effect of financialization and speculation on volatility}
\label{tab:fin}
\begin{tabular}{@{}lll@{}}
\toprule
\textbf{References}                   & \textbf{Proxy used for financialization or speculation}          & \textbf{Impact on volatility}      \\ \midrule
\citet{chang1997interday}  & CFTC’s definition of speculators                                 & \multirow{4}{*}{\textbf{Positive}} \\
\citet{daigler1999impact}              & CFTC’s definition of speculators                                 &                                    \\
\citet{irwin2004effect}                 & Set speculators                                                  &                                    \\
\citet{tang2012index}                  & Commodity index trader (CIT) positions                           &                                    \\ \midrule
\citet{irwin1987note}             & Amount of money invested in traded futures funds                 & \multirow{4}{*}{\textbf{Neutral}}  \\
\citet{irwin1999managed}            & Trading volume of large-commodity pool operators                 &                                    \\
\citet{bryant2006causality}     & CFTC’s definition of speculators                                 &                                    \\
\citet{haigh2007hedge} & Number and positions of commodity pool operators and hedge funds &                                    \\ \midrule
\citet{brunetti2016speculators} & The net positions of hedge funds and floor brokers & \multirow{2}{*}{\textbf{Negative}} \\
\citet{aulerich2012bubbles}     & Commodity index trader (CIT) positions                           &                                    \\ \bottomrule
\end{tabular}
\end{table}
\end{landscape}

\begin{table}[] 
\begin{center}
\caption{Details of the macroeconomics announcements used in the study}
\label{tab:stat1}
\begin{tabular}{@{}lcccc@{}}
\toprule
\multicolumn{1}{c}{\textbf{Announcement}} & \textbf{Frequency} & \textbf{Source*} & \textbf{Unit}     & \textbf{Time} \\ \midrule
\textbf{GDP advance}                      & Quarterly          & BEA             & \%                & 8:30          \\
\textbf{GDP preliminary}                  & Quarterly          & BEA             & \%                & 8:30          \\
\textbf{GDP final}                        & Quarterly          & BEA             & \%                & 8:30          \\
\textbf{Personal income}                  & Monthly            & BEA             & \%                & 8:30          \\
\textbf{ADP employment}                   & Monthly            & ADP             & Number of jobs    & 8:15          \\
\textbf{Initial jobless claims}           & Weekly             & ETA             & Number of claims  & 8:30          \\
\textbf{Non-farm employment}              & Monthly            & BLS             & Number of jobs    & 8:30          \\
\textbf{Factory orders}                   & Monthly            & BC              & \%                & 10:00         \\
\textbf{Industrial production}            & Monthly            & FRB             & \%                & 9:15          \\
\textbf{Construction spending}            & Monthly            & BC              & \%                & 10:00         \\
\textbf{Durable goods orders}             & Monthly            & BC              & \%                & 8:30          \\
\textbf{Advance retail sales}             & Monthly            & BC              & \%                & 8:30          \\
\textbf{Consumer credit}                  & Monthly            & FRB             & USD               & 15:00         \\
\textbf{Personal consumption}             & Monthly            & BEA             & \%                & 8:30          \\
\textbf{Building permits}                 & Monthly            & BC              & Number of permits & 8:30          \\
\textbf{Existing home sales}              & Monthly            & NAR             & Number of homes   & 10:00         \\
\textbf{Housing starts}                   & Monthly            & BC              & Number of homes   & 8:30          \\
\textbf{New home sales}                   & Monthly            & BC              & Number of homes   & 10:00         \\
\textbf{Pending home sales}               & Monthly            & NAR             & \%                & 10:00         \\
\textbf{Trade balance}                    & Monthly            & BEA             & USD               & 8:30          \\
\textbf{Consumer price index}             & Monthly            & BLS             & \%                & 8:30          \\
\textbf{Producer price index}             & Monthly            & BLS             & \%                & 8:30          \\
\textbf{CB Consumer confidence index}     & Monthly            & CB              & Index             & 10:00         \\
\textbf{UM Consumer sentiment}            & Monthly            & TR/UM           & Index             & 9:55          \\ \bottomrule
\end{tabular}
\end{center}
\begin{tablenotes}
        \singlespacing
        \footnotesize
Shows the category, frequency, source, unit of measure, and release time for each macroeconomic announcements.  *(Automatic Data Processing, Inc. (ADP), Bureau of the Census (BC), Bureau of Economic Analysis (BEA), Bureau of Labor Statistics (BLS), Conference Board (CB), Employment and Training Administration (ETA), Federal Reserve Board (FRB), Institute for Supply Management (ISM), National Association of Realtors (NAR), Thomson Reuters/University of Michigan (TR/UM), and U.S. Department of the Treasury (USDT).)
\end{tablenotes}
\end{table}


\begin{landscape}
\begin{table}[]
\begin{center}
\caption{Descriptive statistics: Standardized surprises for each of the macroeconomic announcements.}
\label{tab:stat2}
\begin{tabular}{@{}lccccccc@{}}
\toprule
\multicolumn{1}{c}{\textbf{Announcements}} & \textbf{Nb. Obs.} & \textbf{Min.} & \textbf{1st Qu.} & \textbf{Med.} & \textbf{Mean} & \textbf{3rd Qu.} & \textbf{Max.} \\ \midrule
\textbf{Initial jobless claims}            & 672                      & -3.117        & -0.059           & 0.000           & 0.076         & 0.066            & 20.746        \\
\textbf{ADP Employment}                    & 154                      & -2.549        & -0.040           & 0.005           & 0.054         & 0.070            & 11.931        \\
\textbf{CB Consumer}                       & 155                      & -2.590        & -0.638           & -0.018          & 0.053         & 0.821            & 2.371         \\
\textbf{Advance retail sales}              & 154                      & -4.458        & -0.304           & -0.051          & -0.025        & 0.203            & 9.828         \\
\textbf{Building permit}                   & 155                      & -2.065        & -0.589           & 0.000           & 0.117         & 0.743            & 3.496         \\
\textbf{Construction spending}             & 155                      & -2.884        & -0.698           & -0.093          & -0.155        & 0.372            & 4.093         \\
\textbf{Consumer\_credit}                  & 154                      & -6.213        & -0.511           & 0.116           & 0.010         & 0.588            & 1.741         \\
\textbf{Consumer price index}              & 155                      & -3.799        & -0.760           & 0.000           & -0.152        & 0.000            & 3.039         \\
\textbf{Durable goods orders}              & 161                      & -3.479        & -0.467           & 0.042           & 0.051         & 0.551            & 6.406         \\
\textbf{Existing home sales}               & 155                      & -4.586        & -0.449           & 0.000           & 0.007         & 0.598            & 2.393         \\
\textbf{Factory orders}                    & 153                      & -2.978        & -0.496           & 0.000           & 0.015         & 0.496            & 2.647         \\
\textbf{GDP}                               & 153                      & -2.861        & -0.204           & 0.000           & 0.088         & 0.409            & 4.292         \\
\textbf{Housing starts}                    & 154                      & -2.389        & -0.708           & -0.028          & -0.037        & 0.514            & 3.556         \\
\textbf{Industrial production}             & 291                      & -4.486        & -0.641           & -0.214          & -0.186        & 0.427            & 2.563         \\
\textbf{Michigan Sentiment Index}          & 155                      & -3.271        & -0.414           & 0.075           & -0.021        & 0.508            & 2.519         \\
\textbf{New home sales}                    & 154                      & -3.374        & -0.394           & -0.066          & 0.018         & 0.482            & 2.914         \\
\textbf{Non-farm employment}               & 155                      & -0.690        & -0.053           & -0.001          & 0.094         & 0.056            & 12.069        \\
\textbf{Pending home sales}                & 155                      & -3.905        & -0.413           & 0.000           & 0.034         & 0.558            & 5.669         \\
\textbf{Personal consumption}              & 154                      & -9.063        & -0.363           & 0.000           & -0.131        & 0.363            & 2.538         \\
\textbf{Personal income}                   & 154                      & -4.828        & -0.241           & 0.000           & -0.036        & 0.241            & 10.380        \\
\textbf{Producer price index}              & 155                      & -3.138        & -0.571           & 0.000           & 0.011         & 0.571            & 2.853         \\
\textbf{Trade balance}                     & 154                      & -3.010        & -0.417           & -0.027          & 0.036         & 0.466            & 3.878         \\ \bottomrule
\end{tabular}
\end{center}
\begin{tablenotes}
        \singlespacing
        \footnotesize
In this table, we present some descriptive statistics of the standardized surprise for each of the macroeconomic announcements. The column (Nb. Observations) shows the number of individual surprises that can be calculated over the whole analysis period. The columns (Min.), (1st Qu.), (Median), (Mean), (3rd Qu.) and (Max) present respectively the minimum value, the first quartile, the median, the mean, the third quartile and the maximum value for the standardized surprise of each macroeconomic announcement 
\end{tablenotes}
\end{table}
\end{landscape}

\begin{landscape}
\begin{table}[]
\begin{center}
\caption{Details of the commodity futures contracts in the study}
\label{tab:stat3}
\begin{tabular}{lclll}
\hline
\multicolumn{1}{c}{\textbf{Commodity name}} & \textbf{Commodity Ticker} & \multicolumn{1}{c}{\textbf{Commodity Exchange}} & \multicolumn{1}{c}{\textbf{Price quotation}} & \multicolumn{1}{c}{\textbf{Contract unit}} \\ \hline
\textbf{Crude Oil}                          & CL                        & New York Mercantile Exchange                    & U.S. dollars and cents per barrel            & 1,000 barrels                              \\
\textbf{Gold}                               & GC                        & Commodity Exchange Inc.                         & U.S. dollars and cents per troy ounce        & 100 troy ounces                            \\
\textbf{Copper}                             & HG                        & Commodity Exchange Inc.                         & U.S. dollars and cents per pound             & 25,000 pounds                              \\
\textbf{Natural Gas}                        & NG                        & New York Mercantile Exchange                    & U.S. dollars and cents per MMBtu             & 10,000 MMBtu                               \\
\textbf{Palladium}                          & PA                        & New York Mercantile Exchange                    & U.S. dollars and cents per troy ounce        & 100 troy ounces                            \\
\textbf{Silver}                             & SI                        & Commodity Exchange Inc.                         & U.S. dollars and cents per troy ounce        & 5,000 troy ounces                          \\ \hline
\end{tabular}
\end{center}
\begin{tablenotes}
        \singlespacing
        \footnotesize
This table presents information about the futures contracts of the 6 selected commodities: Crude Oil, Gold, (High-Grade) Copper, Natural Gas, Palladium and Silver. For each commodity, the commodity ticker, commodity exchange, price quotation and contract unit are presented.
\end{tablenotes}
\end{table}
\end{landscape}
%%%%%%%%%%%%%%%%%%%%%%%%%%%
\begin{landscape}
\begin{table}[]
\begin{center}
\caption{Descriptive statistics: 5-minute intraday futures price returns}
\label{tab:stat4}
\begin{tabular}{@{}lllllll@{}}
\toprule
\textbf{Commodity Futures}  & \textbf{Min (\%)} & \textbf{1st Qu. (\%)} & \textbf{Med. (\%)} & \textbf{Mean (\%)} & \textbf{3rd Qu. (\%)} & \textbf{Max (\%)} \\ \midrule
\textbf{Crude Oil (CL=F)}   & -33.9081          & -0.0441               & 0.00               & 0.00             & 0.0447                & 41.64134        \\
\textbf{Gold (GC=F)}        & -2.7822           & -0.0241               & 0.00               & 0.0001             & 0.0244                & 3.0641         \\
\textbf{Copper (HG=F)}      & -4.534           & -0.0363               & 0.00               & -0.0001            & 0.0365                & 8.877         \\
\textbf{Natural Gas (NG=F)} & -6.735           & -0.0528               & 0.00               & -0.0003            & 0.0532                & 15.62        \\
\textbf{Palladium (PA=F)}   & -13.35          & -0.034               & 0.00               & 0.0001             & 0.0348                & 9.467         \\
\textbf{Silver (SI=F)}      & -7.504           & -0.0394               & 0.00               & 0.0001             & 0.0415                & 4.242         \\ \bottomrule
\end{tabular}
\end{center}
\begin{tablenotes}
        \singlespacing
        \footnotesize
Shows descriptive statistics of the 5-minute intraday returns, for each commodity futures. The columns (Min.), (1st Qu.), (Median), (Mean), (3rd Qu.) and (Max) present respectively the minimum value, the first quartile, the median, the mean, the third quartile and the maximum value for the 5 minute intraday returns.
\end{tablenotes}
\end{table}
\end{landscape}



\begin{landscape}
\begin{table}[]
\begin{center}
\caption{Descriptive statistics: MSCT, NLS and Working's T financialization variables}
\label{tab:stat5}
\begin{tabular}{@{}lllllll@{}}
\toprule
                 & \textbf{CL} & \textbf{GC} & \textbf{HG} & \textbf{SI} & \textbf{PA} & \textbf{NG} \\ \midrule
\multicolumn{7}{c}{\textbf{MSCT}}                                                                    \\ \midrule
\textbf{Min.}    & 0.062945    & 0.165982    & 0.122262    & 0.155190    & 0.058447    & 0.037411    \\
\textbf{1st Qu.} & 0.147171    & 0.270250    & 0.234365    & 0.207841    & 0.362742    & 0.129146    \\
\textbf{Median}  & 0.165226    & 0.305187    & 0.282629    & 0.252736    & 0.406176    & 0.234401    \\
\textbf{Mean}    & 0.164179    & 0.304460    & 0.291516    & 0.260574    & 0.391650    & 0.216984    \\
\textbf{3rd Qu.} & 0.185501    & 0.341228    & 0.353014    & 0.307877    & 0.438621    & 0.272414    \\
\textbf{Max.}    & 0.243133    & 0.450365    & 0.496837    & 0.459285    & 0.595550    & 0.402280    \\ \midrule
\multicolumn{7}{c}{\textbf{NLS}}                                                                     \\ \midrule
\textbf{Min.}    & -0.112938   & -0.082052   & -0.323828   & -0.136418   & -0.350847   & -0.274505   \\
\textbf{1st Qu.} & 0.035834    & 0.228504    & -0.106675   & 0.154751    & 0.318095    & -0.173095   \\
\textbf{Median}  & 0,108400    & 0,330220    & 0,017086    & 0,245223    & 0,453227    & -0,081458   \\
\textbf{Mean}    & 0,110749    & 0,309760    & 0,020385    & 0,242498    & 0,421290    & -0,094391   \\
\textbf{3rd Qu.} & 0.183409    & 0.401171    & 0.141731    & 0.333416    & 0.565189    & -0.032108   \\
\textbf{Max.}    & 0.294144    & 0.526856    & 0.441333    & 0.574770    & 0.734284    & 0.079358    \\ \midrule
\multicolumn{7}{c}{\textbf{Working-T}}                                                               \\ \midrule
\textbf{Min.}    & 1.021866    & 1.043735    & 1.024398    & 1.014335    & 1.000000    & 1.004389    \\
\textbf{1st Qu.} & 1.076590    & 1.090740    & 1.152226    & 1.076194    & 1.106552    & 1.131993    \\
\textbf{Median}  & 1.100555    & 1.140149    & 1.234470    & 1.117848    & 1.155713    & 1.209708    \\
\textbf{Mean}    & 1.103168    & 1.160994    & 1.254092    & 1.150930    & 1.193110    & 1.237466    \\
\textbf{3rd Qu.} & 1.124351    & 1.197429    & 1.356929    & 1.188077    & 1.234792    & 1.341500    \\
\textbf{Max.}    & 1.247876    & 1.663823    & 1.655362    & 1.604836    & 1.938426    & 1.558946    \\ \bottomrule
\end{tabular}
\end{center}
\begin{tablenotes}
        \singlespacing
        \footnotesize
Shows descriptive statistics of the financialization variables, for each commodity futures. The line (Min.), (1st Qu.), (Median), (Mean), (3rd Qu.) and (Max) present respectively the minimum value, the first quartile, the median, the mean, the third quartile and the maximum value for the 5-minute intraday returns.
\end{tablenotes}
\end{table}

\end{landscape}

\begin{landscape}
\begin{table}[]
\caption{Announcement and financialization effects on futures returns}
\label{tab:reg_return}
\resizebox{\columnwidth}{!}{%
\begin{tabular}{@{}lllllllllllll@{}}
\toprule
\textbf{Commodities}              & \multicolumn{2}{c}{\textbf{Crude Oil}}    & \multicolumn{2}{c}{\textbf{Gold}}         & \multicolumn{2}{c}{\textbf{Copper}}       & \multicolumn{2}{c}{\textbf{Natural Gas}}  & \multicolumn{2}{c}{\textbf{Palladium}}    & \multicolumn{2}{c}{\textbf{Silver}}       \\ \midrule
\textbf{Announcements}            & \textbf{$\gamma_m$} & \textbf{$\theta_m$} & \textbf{$\gamma_m$} & \textbf{$\theta_m$} & \textbf{$\gamma_m$} & \textbf{$\theta_m$} & \textbf{$\gamma_m$} & \textbf{$\theta_m$} & \textbf{$\gamma_m$} & \textbf{$\theta_m$} & \textbf{$\gamma_m$} & \textbf{$\theta_m$} \\ \midrule
\textbf{Initial jobless claims}   & -0.002***           & 0.010***            & 0.007***            & -0.013***           & -0,0001             & -0,001              & 0,0001              & 0,0004              & 0,0001              & 0,0004              & 0.005***            & -0.022***           \\
\textbf{ADP Employment}           & 0.007***            & -0.026***           & -0.017***           & 0.035***            & 0.0003***           & 0,001               & -0.001***           & -0.034***           & -0.001***           & -0.034***           & -0.006***           & 0.027***            \\
\textbf{CB Consumer}              & 0.001***            & -0.005***           & -0.0004***          & 0,0005              & 0,0001              & 0,0004              & 0,0003              & -0,00002            & 0,0003              & -0,00002            & -0.001***           & 0,001               \\
\textbf{Advance retail sales}     & 0.002***            & -0.008***           & -0.003***           & 0.005***            & 0,0001              & -0,001              & -0,0002             & -0,004              & -0,0002             & -0,004              & -0.001***           & 0.005***            \\
\textbf{Building permit}          & -0,0001             & 0,001               & -0.0005***          & 0.001***            & 0,0001              & -0,0002             & 0.001**             & 0.004**             & 0.001**             & 0.004**             & -0.001***           & 0.001**             \\
\textbf{Construction spending}    & 0,0001              & -0,001              & -0.0003*            & 0,001               & -0,00004            & -0.002***           & -0,0003             & -0,001              & -0,0003             & -0,001              & -0,0003             & 0,001               \\
\textbf{Consumer credit}          & -0,0001             & 0,001               & -0,0001             & 0,0003              & 0,00002             & 0,0001              & -0,0001             & -0,001              & -0,0001             & -0,001              & -0,0001             & 0,001               \\
\textbf{Consumer price index}     & -0.0005**           & 0,001               & -0.001***           & 0.003***            & -0,0001             & -0.002***           & 0,0004              & 0,002               & 0,0004              & 0,002               & -0.001***           & 0.004***            \\
\textbf{Durable goods orders}     & 0.002***            & -0.008***           & -0.001***           & 0.002***            & 0.0001**            & 0,0002              & 0,00002             & -0,001              & 0,00002             & -0,001              & -0.001***           & 0,001               \\
\textbf{Existing home sales}      & 0.001***            & -0.006***           & -0,0002             & 0,001               & 0.0003***           & -0,001              & 0,001               & 0,003               & 0,001               & 0,003               & -0,0003             & 0,001               \\
\textbf{Factory orders}           & 0,0002              & -0,001              & -0,0001             & -0,0003             & 0,00003             & -0,0001             & 0.001***            & 0.006**             & 0.001***            & 0.006**             & -0,0003             & 0,0003              \\
\textbf{GDP}                      & 0,0003              & -0,001              & -0.001***           & 0.001***            & 0.0003***           & 0,0002              & -0,0003             & -0.003*             & -0,0003             & -0.003*             & -0.002***           & 0.003***            \\
\textbf{Housing starts}           & 0.001*              & -0,002              & -0.001***           & 0.001***            & 0,00002             & 0,001               & -0,0003             & 0,0005              & -0,0003             & 0,0005              & -0.001***           & 0,001               \\
\textbf{Industrial production}    & 0,0003              & -0,001              & -0,00001            & -0,001              & -0,0001             & 0,0001              & 0,0002              & 0,002               & 0,0002              & 0,002               & -0,0001             & -0,001              \\
\textbf{Michigan Sentiment Index} & 0,0003              & -0,001              & -0,0002             & -0,0002             & 0,0001              & 0,0005              & -0,0003             & -0,002              & -0,0003             & -0,002              & -0.0004***          & 0,0003              \\
\textbf{New home sales}           & 0.001***            & -0.004**            & -0.001***           & 0.001**             & 0.0003***           & -0.001**            & 0,00003             & -0,001              & 0,00003             & -0,001              & -0.0004*            & -0,0003             \\
\textbf{Non-farm employment}      & 0.035***            & -0.132***           & -0.045***           & 0.098***            & -0,0001             & -0,001              & -0.003***           & -0.088***           & -0.003***           & -0.088***           & -0.009***           & 0.040***            \\
\textbf{Pending home sales}       & 0.001***            & -0.004***           & -0,0002             & 0,0001              & 0.0003***           & -0,001              & -0,00002            & -0,003              & -0,00002            & -0,003              & -0,00001            & -0,001              \\
\textbf{Personal consumption}     & 0,0002              & -0,0005             & -0.0003**           & 0.001**             & 0,00001             & 0,0001              & -0,0001             & 0,0001              & -0,0001             & 0,0001              & 0.0002*             & -0,001              \\
\textbf{Personal income}          & 0.006***            & -0.026***           & -0.008***           & 0.017***            & -0.0003**           & -0.002*             & -0,0005             & -0.011***           & -0,0005             & -0.011***           & -0.002***           & 0.010***            \\
\textbf{Producer price index}     & 0.0004**            & -0.003**            & -0.001***           & 0.002***            & -0,00004            & -0,0004             & -0,0004             & -0,002              & -0,0004             & -0,002              & -0,0002             & -0,0001             \\
\textbf{Trade balance}            & -0,0001             & 0,0004              & -0.001***           & 0.002***            & -0,00004            & -0,001              & -0,0001             & -0,002              & -0,0001             & -0,002              & -0,0003             & 0,002               \\ \midrule
\textbf{$R^2$}                    & \multicolumn{2}{c}{0,001}                 & \multicolumn{2}{c}{0,002}                 & \multicolumn{2}{c}{0,0003}                & \multicolumn{2}{c}{0,0004}                & \multicolumn{2}{c}{0,0004}                & \multicolumn{2}{c}{0,001}                 \\
\textbf{Observations}             & \multicolumn{2}{c}{971826}                & \multicolumn{2}{c}{968141}                & \multicolumn{2}{c}{916716}                & \multicolumn{2}{c}{880054}                & \multicolumn{2}{c}{880054}                & \multicolumn{2}{c}{959102}                \\ \bottomrule
\end{tabular}%
}

\begin{tablenotes}
        \singlespacing
        \footnotesize
Presents the estimates of eq. \ref{eq:Model 1}, using the method proposed by \citep{kurov2019price} and financialization variable $X_{t,2}=NLS_t$. The $\gamma_m$ coefficients capture the instantaneous change in return when an announcement has just occurred and especially if that announcement was unanticipated. The coefficients $\theta_m$ capture the instantaneous change in return when an announcement has just occurred in conjunction with the level of financialization.

    \end{tablenotes}
\end{table}
\end{landscape}

\begin{landscape}
\begin{table}[]
\caption{ Announcement and financialization effects on futures conditional variance}
\label{tab:reg_vol}
\resizebox{\columnwidth}{!}{%
\begin{tabular}{@{}lllllllllllll@{}}
\toprule
\textbf{Commodities}              & \multicolumn{2}{c}{\textbf{Crude Oil}} & \multicolumn{2}{c}{\textbf{Gold}}     & \multicolumn{2}{c}{\textbf{Copper}}   & \multicolumn{2}{c}{\textbf{Natural Gas}} & \multicolumn{2}{c}{\textbf{Palladium}} & \multicolumn{2}{c}{\textbf{Silver}}   \\ \midrule
\textbf{Announcements}            & \textbf{$\Phi_m$}  & \textbf{$\phi_m$} & \textbf{$\Phi_m$} & \textbf{$\phi_m$} & \textbf{$\Phi_m$} & \textbf{$\phi_m$} & \textbf{$\Phi_m$}   & \textbf{$\phi_m$}  & \textbf{$\Phi_m$}  & \textbf{$\phi_m$} & \textbf{$\Phi_m$} & \textbf{$\phi_m$} \\ \midrule
\textbf{Initial jobless claims}   & 0.001***           & -0.003***         & 0.0003***         & 0,0002            & 0.0002***         & -0,0003           & -0.0003***          & -0,0003            & 0.001***           & -0.002***         & 0.001***          & -0.001***         \\
\textbf{ADP Employment}           & 0,0003             & -0,002            & 0.001***          & -0.001***         & 0.0002***         & -0,00002          & 0,0002              & 0,0005             & -0.001**           & 0.001*            & 0.001***          & -0,0002           \\
\textbf{CB Consumer}              & 0.001***           & -0.004**          & 0.0003**          & -0,0002           & 0.0003***         & -0.001**          & -0,00002            & 0,0001             & 0,0003             & -0,0003           & 0.0004**          & 0                 \\
\textbf{Advance retail sales}     & 0.002***           & -0.008***         & 0.001***          & -0.001*           & 0.0003***         & -0.002***         & 0.001***            & 0.003**            & -0,0003            & 0,001             & 0.001***          & 0,001             \\
\textbf{Building permit}          & 0,0002             & -0,003            & 0.001**           & -0.002**          & -0,0001           & 0,0004            & 0,0001              & -0,002             & 0.015***           & -0.026***         & 0,001             & -0,003            \\
\textbf{Construction spending}    & 0.001***           & -0.005***         & 0.001***          & -0.001***         & 0.0004***         & -0.002***         & -0.0003*            & -0.004***          & 0.001*             & -0,0003           & 0.001***          & -0,001            \\
\textbf{Consumer credit}          & 0.0005*            & -0,003            & 0,0001            & -0,0001           & 0,0001            & 0,0005            & 0                   & 0,00003            & -0,001             & 0,001             & 0,0002            & -0,0001           \\
\textbf{Consumer price index}     & 0,0002             & -0,001            & 0.001***          & -0,0005           & 0,00003           & 0,0002            & 0,00002             & 0,0003             & -0,0003            & 0,001             & 0.001***          & -0,0001           \\
\textbf{Durable goods orders}     & 0.001***           & -0.003**          & 0,0001            & 0                 & -0,00001          & -0,0003           & -0,00003            & -0,001             & 0,0004             & -0,0001           & -0,00001          & 0,001             \\
\textbf{Existing home sales}      & 0,0004             & 0,00004           & 0.0002*           & 0,0004            & 0.0002***         & -0.002***         & 0,0001              & -0,001             & 0.002***           & -0.003***         & 0.0004**          & -0,0002           \\
\textbf{Factory orders}           & 0.001***           & -0.003**          & -0,0001           & 0.001***          & 0.0002***         & -0.001***         & -0.0004**           & -0.006***          & 0.001***           & -0.002**          & -0,0001           & 0.002**           \\
\textbf{GDP}                      & 0.001***           & -0.005***         & 0.0003**          & 0,0003            & 0.0002***         & 0,0003            & -0.0005**           & -0.003*            & 0.001**            & -0.001*           & 0.001***          & -0,0002           \\
\textbf{Housing starts}           & 0,001              & -0,002            & -0.001*           & 0.002*            & 0,0002            & -0,001            & -0,0002             & 0,001              & -0.015***          & 0.026***          & -0,0002           & 0,001             \\
\textbf{Industrial production}    & 0.001***           & -0.008***         & 0.0002**          & 0,00002           & 0,0001            & -0,001            & -0,0003             & -0,001             & 0.001***           & -0.002***         & 0.0003*           & -0,0001           \\
\textbf{Michigan Sentiment Index} & 0,0003             & -0,001            & -0.0002**         & 0.001***          & 0,00002           & 0,0002            & -0,0001             & 0,001              & 0,0002             & -0,0001           & 0,00005           & 0.001*            \\
\textbf{New home sales}           & 0.001***           & -0.005***         & 0.0004***         & 0,0001            & 0.0002***         & 0,0003            & -0,0002             & -0,001             & 0.001***           & -0.002***         & 0.0004**          & 0,001             \\
\textbf{Non-farm employment}      & 0.004***           & -0.019***         & 0.004***          & -0.004***         & 0.001***          & -0.006***         & -0,0005             & -0.015***          & 0.002***           & -0,001            & 0.003***          & 0,001             \\
\textbf{Pending home sales}       & 0.001**            & -0.003*           & 0.0002*           & -0,0003           & 0.0003***         & 0,0003            & -0,0002             & -0.003*            & 0.001***           & -0.002**          & 0,0002            & 0,001             \\
\textbf{Personal consumption}     & -0,0003            & 0,002             & 0,0001            & 0,0001            & -0,0001           & 0,0004            & -0,0004             & -0.005***          & 0.001*             & -0,001            & -0,00002          & 0,001             \\
\textbf{Personal income}          & 0,001              & 0,004             & -0.001***         & 0.005***          & 0.0005***         & 0.006***          & 0.001***            & 0.014***           & -0.001*            & 0.002*            & 0.001**           & -0.004***         \\
\textbf{Producer price index}     & 0,0002             & 0,0002            & -0,0001           & 0.001**           & -0,00004          & 0,0003            & 0                   & -0,001             & 0.001**            & -0,0003           & 0.001***          & -0,001            \\
\textbf{Trade balance}            & 0,0004             & -0,0001           & -0.001***         & 0.004***          & -0.0002***        & 0,001             & 0,0002              & 0,002              & 0,001              & -0,001            & -0.001***         & 0.006***          \\ \bottomrule
\end{tabular}%
}

\begin{tablenotes}
        \singlespacing
        \footnotesize
        Presents the estimate of eq. \ref{eq:Variance} using financialization variable $X_{2,t}=NLS_t$. The $\Phi_m$ coefficients capture the instantaneous change in the conditional variance when an announcement has just occurred. The $\phi_m$ coefficients capture the conditional variance when an announcement has just occurred in conjunction with the level of financialization.

    \end{tablenotes}
    
\end{table}
\end{landscape}


\begin{landscape}
\begin{table}[]
\caption{Announcement and financialization effects on futures returns: Results based on a financialization variable built using money manager positions }
\label{tab:reg_ret_mm}
\resizebox{\columnwidth}{!}{%
\begin{tabular}{@{}lllllllllllll@{}}
\toprule
\textbf{Commodities}              & \multicolumn{2}{c}{\textbf{Crude Oil}}                                            & \multicolumn{2}{c}{\textbf{Gold}}                                                 & \multicolumn{2}{c}{\textbf{Copper}}                                               & \multicolumn{2}{c}{\textbf{Natural Gas}}                                          & \multicolumn{2}{c}{\textbf{Palladium}}                                            & \multicolumn{2}{c}{\textbf{Silver}}                                               \\ \midrule
\textbf{Announcements}            & \multicolumn{1}{c}{\textbf{$\gamma_m$}} & \multicolumn{1}{c}{\textbf{$\theta_m$}} & \multicolumn{1}{c}{\textbf{$\gamma_m$}} & \multicolumn{1}{c}{\textbf{$\theta_m$}} & \multicolumn{1}{c}{\textbf{$\gamma_m$}} & \multicolumn{1}{c}{\textbf{$\theta_m$}} & \multicolumn{1}{c}{\textbf{$\gamma_m$}} & \multicolumn{1}{c}{\textbf{$\theta_m$}} & \multicolumn{1}{c}{\textbf{$\gamma_m$}} & \multicolumn{1}{c}{\textbf{$\theta_m$}} & \multicolumn{1}{c}{\textbf{$\gamma_m$}} & \multicolumn{1}{c}{\textbf{$\theta_m$}} \\ \midrule
\textbf{Initial jobless claims}   & -0,0003                                 & 0,0002                                  & 0.003***                                & -0.009***                               & -0.0002**                               & -0,0010                                 & 0,0001                                  & 0,0010                                  & -0.001***                               & 0.007***                                & -0,0010                                 & 0.005**                                 \\
\textbf{ADP Employment}           & 0.007***                                & -0.033***                               & 0,0002                                  & -0,0020                                 & 0.0003***                               & 0,0010                                  & -0.0003*                                & -0,0100                                 & 0,0003                                  & -0.003*                                 & -0,0003                                 & -0,0020                                 \\
\textbf{CB Consumer}              & 0.001***                                & -0.006*                                 & -0.0003***                              & -0,0001                                 & 0,0001                                  & 0,0003                                  & 0.0003*                                 & 0,0010                                  & -0.001*                                 & 0,0010                                  & -0.0005***                              & 0,0010                                  \\
\textbf{Advance retail sales}     & 0.002***                                & -0.010***                               & -0.001***                               & 0,0001                                  & 0,0001                                  & -0,0010                                 & 0,0000                                  & -0,0010                                 & 0,0001                                  & -0,0010                                 & -0.001***                               & 0.006***                                \\
\textbf{Building permit}          & -0,0001                                 & 0,0010                                  & -0.0003***                              & 0.001*                                  & 0,0001                                  & -0,0001                                 & 0,0002                                  & 0,0010                                  & 0,0003                                  & -0.001**                                & -0.0004***                              & 0.001*                                  \\
\textbf{Construction spending}    & 0,0003                                  & -0,0030                                 & -0.0004***                              & 0.001***                                & 0,0000                                  & -0.001**                                & -0,0001                                 & -0,0010                                 & 0,0010                                  & -0,0010                                 & -0,0003                                 & 0,0010                                  \\
\textbf{Consumer credit}          & -0,0001                                 & 0,0010                                  & -0,0001                                 & 0,0003                                  & 0,0000                                  & 0,0001                                  & 0,0000                                  & -0,0004                                 & 0,0003                                  & -0,0010                                 & -0,0001                                 & 0,0010                                  \\
\textbf{Consumer price index}     & -0.001**                                & 0,0020                                  & -0.001***                               & 0.002***                                & 0,0000                                  & -0.001*                                 & 0,0002                                  & 0,0010                                  & 0.001***                                & -0.002***                               & -0.001***                               & 0.003***                                \\
\textbf{Durable goods orders}     & 0.001***                                & -0.007***                               & -0.001***                               & 0.002***                                & 0.0002**                                & -0,0004                                 & 0,0001                                  & 0,0003                                  & -0,0005                                 & 0.002**                                 & -0.0005***                              & 0,0010                                  \\
\textbf{Existing home sales}      & 0.001***                                & -0.007**                                & -0,0001                                 & 0,0003                                  & 0.0003***                               & -0,0001                                 & 0,0001                                  & 0,0010                                  & -0,0010                                 & 0,0010                                  & -0,0001                                 & 0,0010                                  \\
\textbf{Factory orders}           & -0,0003                                 & 0,0020                                  & -0,0001                                 & -0,0010                                 & 0,0001                                  & -0,0002                                 & 0.0004*                                 & 0,0030                                  & 0,0000                                  & -0,0003                                 & -0,0003                                 & 0,0004                                  \\
\textbf{GDP}                      & 0,0004                                  & -0,0020                                 & -0.001***                               & -0,0005                                 & 0.0003***                               & 0,0001                                  & 0,0000                                  & -0.003*                                 & -0.002***                               & 0.004***                                & -0.001***                               & 0,0010                                  \\
\textbf{Housing starts}           & -0,0001                                 & 0,0020                                  & -0.0003***                              & 0,0002                                  & 0,0000                                  & 0,0010                                  & -0.0003**                               & 0,0010                                  & -0.0004*                                & 0.001*                                  & -0.0004***                              & 0,0005                                  \\
\textbf{Industrial production}    & 0,0000                                  & 0,0004                                  & -0,0002                                 & -0,0002                                 & 0,0000                                  & -0,0003                                 & 0,0001                                  & 0.003**                                 & -0.001*                                 & 0,0010                                  & -0,0002                                 & -0,0010                                 \\
\textbf{Michigan Sentiment Index} & 0,0003                                  & -0,0010                                 & -0.0002***                              & 0,0000                                  & 0,0000                                  & 0,0003                                  & -0,0001                                 & -0,0010                                 & -0.001**                                & 0.001*                                  & -0.0004***                              & 0,0002                                  \\
\textbf{New home sales}           & 0,0010                                  & -0,0030                                 & -0.0005***                              & 0.001*                                  & 0.0003***                               & -0,0010                                 & 0,0001                                  & -0,0010                                 & -0,0005                                 & 0,0010                                  & -0.0004**                               & -0,0003                                 \\
\textbf{Non-farm employment}      & 0.022***                                & -0.117***                               & 0,0002                                  & -0,0030                                 & -0,0001                                 & 0,0002                                  & -0,0002                                 & -0.030*                                 & 0.001*                                  & -0.006*                                 & -0.003**                                & 0.019**                                 \\
\textbf{Pending home sales}       & 0.001***                                & -0.007**                                & -0,0001                                 & -0,0003                                 & 0.0003***                               & -0,0005                                 & 0,0002                                  & -0,0020                                 & -0,0003                                 & 0,0010                                  & -0,0001                                 & -0,0004                                 \\
\textbf{Personal consumption}     & -0.001*                                 & 0.006**                                 & -0,0001                                 & 0,0003                                  & 0,0000                                  & 0,0003                                  & -0,0001                                 & 0,0010                                  & 0,0004                                  & -0.001*                                 & 0,0001                                  & -0,0010                                 \\
\textbf{Personal income}          & 0.004***                                & -0.027***                               & -0.002***                               & 0.009***                                & -0.0002**                               & -0.002*                                 & 0,0002                                  & -0.011***                               & 0.0005*                                 & -0.002**                                & -0.001**                                & 0,0030                                  \\
\textbf{Producer price index}     & 0,0004                                  & -0.003*                                 & -0.001***                               & 0.002***                                & 0,0000                                  & -0,0010                                 & -0,0002                                 & -0,0010                                 & 0,0000                                  & -0,0003                                 & 0,0000                                  & -0,0010                                 \\
\textbf{Trade balance}            & -0,0001                                 & 0,0010                                  & -0.0004***                              & 0.001**                                 & 0,0000                                  & -0,0004                                 & 0,0001                                  & -0,0010                                 & 0,0001                                  & 0,0000                                  & -0.0003*                                & 0.002**                                 \\ \bottomrule
\end{tabular}%
}

\begin{tablenotes}
        \singlespacing
        \footnotesize
       Presents the estimates of eq. \ref{eq:Model 1}, using the method proposed by \citep{kurov2019price} and financialization variable $X_{t,2}=NLS_t$. Only the positions of the money managers are included in the NLS index. The $\gamma_m$ coefficients capture the instantaneous change in return when an announcement has just occurred and especially if that announcement was unanticipated. The coefficients $\theta_m$ capture the instantaneous change in return when an announcement has just occurred in conjunction with the level of financialization.
For all the regressions estimated in this table, we include the control variable NBER to control for the business cycle. 


    \end{tablenotes}
\end{table}
\end{landscape}

\begin{landscape}
\begin{table}[]
\caption{ Announcement and financialization effects on futures returns: Results based on a financialization variable built using swap dealer positions }
\label{tab:reg_ret_swap}
\resizebox{\columnwidth}{!}{%
\begin{tabular}{@{}lllllllllllll@{}}
\toprule
\textbf{Commodities}              & \multicolumn{2}{c}{\textbf{Crude Oil}}                                            & \multicolumn{2}{c}{\textbf{Gold}}                                                 & \multicolumn{2}{c}{\textbf{Copper}}                                               & \multicolumn{2}{c}{\textbf{Natural Gas}}                                          & \multicolumn{2}{c}{\textbf{Palladium}}                                            & \multicolumn{2}{c}{\textbf{Silver}}                                               \\ \midrule
\textbf{Announcements}            & \multicolumn{1}{c}{\textbf{$\gamma_m$}} & \multicolumn{1}{c}{\textbf{$\theta_m$}} & \multicolumn{1}{c}{\textbf{$\gamma_m$}} & \multicolumn{1}{c}{\textbf{$\theta_m$}} & \multicolumn{1}{c}{\textbf{$\gamma_m$}} & \multicolumn{1}{c}{\textbf{$\theta_m$}} & \multicolumn{1}{c}{\textbf{$\gamma_m$}} & \multicolumn{1}{c}{\textbf{$\theta_m$}} & \multicolumn{1}{c}{\textbf{$\gamma_m$}} & \multicolumn{1}{c}{\textbf{$\theta_m$}} & \multicolumn{1}{c}{\textbf{$\gamma_m$}} & \multicolumn{1}{c}{\textbf{$\theta_m$}} \\ \midrule
\textbf{Initial jobless claims}   & -0.002***                               & -0.009***                               & 0.004***                                & 0.011***                                & 0.001**                                 & -0.006**                                & 0,000                                   & 0,001                                   & 0,0003                                  & -0,002                                  & 0,0001                                  & 0,001                                   \\
\textbf{ADP Employment}           & 0.005***                                & 0.023***                                & -0.009***                               & -0.023***                               & -0,001                                  & 0,005                                   & -0.001***                               & 0.032***                                & -0.001**                                & 0.005**                                 & -0.002***                               & -0.034***                               \\
\textbf{CB Consumer}              & 0.001***                                & 0.005***                                & -0.0003***                              & -0,0002                                 & -0.0005**                               & 0.002***                                & 0,0003                                  & -0,001                                  & -0.0002*                                & -0,001                                  & -0.0003***                              & -0,00001                                \\
\textbf{Advance retail sales}     & 0.002***                                & 0.007***                                & -0.002***                               & -0.005***                               & -0.001**                                & 0.004**                                 & -0,0001                                 & 0,003                                   & -0.0003*                                & 0,001                                   & -0.001***                               & -0,001                                  \\
\textbf{Building permit}          & 0,0001                                  & 0,001                                   & -0.0003***                              & -0.001*                                 & 0,0002                                  & -0,001                                  & 0.001***                                & -0.007***                               & -0.0002**                               & 0,001                                   & -0.0002**                               & 0,0002                                  \\
\textbf{Construction spending}    & 0,00002                                 & 0,001                                   & -0,0001                                 & -0,0004                                 & -0.001**                                & 0.002**                                 & 0,0001                                  & -0,002                                  & 0.0003*                                 & 0.002*                                  & -0,0001                                 & -0,001                                  \\
\textbf{Consumer credit}          & 0,000                                   & -0,0001                                 & 0,000                                   & 0,00001                                 & -0,00003                                & 0,0002                                  & -0,0002                                 & 0,002                                   & 0,0001                                  & 0,001                                   & 0,0000                                  & 0,00003                                 \\
\textbf{Consumer price index}     & -0.0005***                              & -0.002**                                & -0.001***                               & -0.004***                               & -0.001***                               & 0.003***                                & 0,0003                                  & -0,002                                  & -0,0001                                 & 0.002**                                 & -0.001***                               & -0.002**                                \\
\textbf{Durable goods orders}     & 0.001***                                & 0.004***                                & -0.0004***                              & -0.001***                               & -0,0001                                 & 0,001                                   & 0,0002                                  & -0,001                                  & 0.0003**                                & 0,0003                                  & -0,0002                                 & 0.004***                                \\
\textbf{Existing home sales}      & 0.0003**                                & 0.003***                                & -0,0001                                 & -0,0001                                 & -0.001***                               & 0.004***                                & 0.001*                                  & -0.004*                                 & -0.0002*                                & -0,001                                  & -0,00002                                & 0,001                                   \\
\textbf{Factory orders}           & 0,0001                                  & 0,001                                   & -0,0002                                 & 0,0002                                  & 0,0001                                  & -0,0004                                 & 0.001**                                 & -0.006**                                & -0,0002                                 & -0,001                                  & -0.0003**                               & -0,001                                  \\
\textbf{GDP}                      & 0.0003**                                & 0,001                                   & -0.001***                               & 0,0001                                  & 0,0001                                  & 0,001                                   & -0,0003                                 & 0.004*                                  & -0.001***                               & -0.001*                                 & -0.001***                               & -0,001                                  \\
\textbf{Housing starts}           & 0.0003*                                 & 0,001                                   & -0.001***                               & -0.002***                               & -0,0002                                 & 0.001*                                  & -0,0003                                 & -0,0001                                 & -0,0001                                 & -0.001**                                & -0.0004***                              & -0.002**                                \\
\textbf{Industrial production}    & 0,0003                                  & 0,001                                   & 0,00004                                 & 0.002***                                & -0,0001                                 & 0,0004                                  & 0,000                                   & -0,001                                  & -0,0002                                 & -0,0004                                 & -0.0003**                               & 0,001                                   \\
\textbf{Michigan Sentiment Index} & 0.0003**                                & 0,0003                                  & -0.0001**                               & 0,001                                   & 0,0002                                  & -0,001                                  & -0,0003                                 & 0,002                                   & -0,0001                                 & -0,001                                  & -0.0003***                              & 0,001                                   \\
\textbf{New home sales}           & 0.001***                                & 0.003***                                & -0.0003***                              & 0,0003                                  & -0.0004**                               & 0.003***                                & 0,0002                                  & -0,0004                                 & -0,0001                                 & -0,001                                  & -0.0004***                              & 0,001                                   \\
\textbf{Non-farm employment}      & 0.022***                                & 0.108***                                & -0.033***                               & -0.090***                               & -0.012***                               & 0.064***                                & -0.002***                               & 0.069***                                & -0.002**                                & 0.010**                                 & -0,001                                  & -0,012                                  \\
\textbf{Pending home sales}       & 0.001***                                & 0.003***                                & -0.0002*                                & -0,0003                                 & -0,0002                                 & 0,002                                   & 0,0002                                  & 0,001                                   & 0,00003                                 & -0,0003                                 & -0,0001                                 & 0,001                                   \\
\textbf{Personal consumption}     & 0,0002                                  & 0,001                                   & -0.0003**                               & -0.001**                                & -0,0004                                 & 0,002                                   & -0,0001                                 & 0,001                                   & -0,00003                                & 0.001**                                 & 0,0001                                  & 0.004***                                \\
\textbf{Personal income}          & 0.003***                                & 0.017***                                & -0.005***                               & -0.015***                               & -0.003***                               & 0.011***                                & -0,0002                                 & 0,006                                   & -0.0005**                               & 0.003***                                & -0.0003**                               & -0.019***                               \\
\textbf{Producer price index}     & 0,0001                                  & 0,001                                   & -0.0002**                               & -0,0002                                 & -0.0005*                                & 0.002*                                  & -0,0002                                 & 0,001                                   & -0,0001                                 & 0,001                                   & -0.0002**                               & 0,001                                   \\
\textbf{Trade balance}            & -0,0001                                 & -0,0001                                 & -0.0003*                                & -0,001                                  & -0,0003                                 & 0,001                                   & 0,0001                                  & 0,0004                                  & -0,00003                                & -0,001                                  & 0,00003                                 & -0.002*                                 \\ \bottomrule
\end{tabular}%
}

\begin{tablenotes}
        \singlespacing
        \footnotesize
       Presents the estimates of eq. \ref{eq:Model 1}, using the method proposed by \citep{kurov2019price} and financialization variable $X_{t,2}=NLS_t$. Only the positions of the swap dealers are included in the NLS index. The $\gamma_m$ coefficients capture the instantaneous change in return when an announcement has just occurred and especially if that announcement was unanticipated. The coefficients $\theta_m$ capture the instantaneous change in return when an announcement has just occurred in conjunction with the level of financialization.
For all the regressions estimated in this table, we include the control variable NBER to control for the business cycle. 


    \end{tablenotes}
    
\end{table}
\end{landscape}


\begin{landscape}
\begin{table}[]
\caption{Announcement and financialization effects on commodity futures conditional variance: Results  based on a financialization variable built using money manager positions }
\label{tab:reg_vol_mm}
\resizebox{\columnwidth}{!}{%
\begin{tabular}{@{}lllllllllllll@{}}
\toprule
\textbf{Commodities}              & \multicolumn{2}{c}{\textbf{Crude Oil}}                                        & \multicolumn{2}{c}{\textbf{Gold}}                                             & \multicolumn{2}{c}{\textbf{Copper}}                                           & \multicolumn{2}{c}{\textbf{Natural Gas}}                                      & \multicolumn{2}{c}{\textbf{Palladium}}                                        & \multicolumn{2}{c}{\textbf{Silver}}                                           \\ \midrule
\textbf{Announcements}            & \multicolumn{1}{c}{\textbf{$\Phi_m$}} & \multicolumn{1}{c}{\textbf{$\phi_m$}} & \multicolumn{1}{c}{\textbf{$\Phi_m$}} & \multicolumn{1}{c}{\textbf{$\phi_m$}} & \multicolumn{1}{c}{\textbf{$\Phi_m$}} & \multicolumn{1}{c}{\textbf{$\phi_m$}} & \multicolumn{1}{c}{\textbf{$\Phi_m$}} & \multicolumn{1}{c}{\textbf{$\phi_m$}} & \multicolumn{1}{c}{\textbf{$\Phi_m$}} & \multicolumn{1}{c}{\textbf{$\phi_m$}} & \multicolumn{1}{c}{\textbf{$\Phi_m$}} & \multicolumn{1}{c}{\textbf{$\phi_m$}} \\ \midrule
\textbf{Initial jobless claims}   & 0.001***                              & -0.003***                             & 0.0003***                             & 0,0002                                & 0.0002***                             & -0,0003                               & -0.0003***                            & -0,001                                & 0.002***                              & -0.002***                             & 0.001***                              & -0.001***                             \\
\textbf{ADP Employment}           & -0,0001                               & 0,0005                                & 0.0005***                             & -0,0005                               & 0.0002***                             & 0,0004                                & 0,0001                                & 0,0005                                & -0.001*                               & 0.001                                 & 0.001***                              & -0,00002                              \\
\textbf{CB Consumer}              & 0.001***                              & -0.007**                              & 0.0002**                              & 0,0001                                & 0.0003***                             & -0,0005                               & -0,00004                              & 0,00004                               & 0.0001                                & 0.00002                               & 0.0004***                             & 0,00004                               \\
\textbf{Advance retail sales}     & 0.001***                              & -0.009***                             & 0.001***                              & -0,0003                               & 0.0003***                             & -0.001***                             & 0.0003***                             & 0.004***                              & -0.0003                               & 0.001                                 & 0.001***                              & 0,001                                 \\
\textbf{Building permit}          & 0,001                                 & -0,013                                & 0.001**                               & -0.002*                               & -0,0002                               & -0,0001                               & 0,001                                 & 0,002                                 & 0.015***                              & -0.029***                             & 0,001                                 & -0,005                                \\
\textbf{Construction spending}    & 0.001***                              & -0.006**                              & 0.001***                              & -0,0004                               & 0.0004***                             & -0.002***                             & -0,00003                              & -0.003***                             & 0.001***                              & -0.001*                               & 0.001***                              & -0,0003                               \\
\textbf{Consumer credit}          & 0,0004                                & -0,003                                & 0,0001                                & 0,0001                                & 0,00004                               & 0,0004                                & 0                                     & 0,0001                                & -0.001                                & 0.001                                 & 0,0001                                & 0,0004                                \\
\textbf{Consumer price index}     & 0,00002                               & -0,0003                               & 0.0005***                             & 0,0004                                & 0,00001                               & 0,0005                                & -0,00003                              & 0,00004                               & -0.001                                & 0.002*                                & 0.001***                              & -0,001                                \\
\textbf{Durable goods orders}     & 0,0003                                & -0,0002                               & 0,0001                                & 0,0001                                & 0,000                                 & -0,0003                               & 0,0001                                & -0,0002                               & 0.0005                                & -0.0004                               & 0,0001                                & 0,0004                                \\
\textbf{Existing home sales}      & 0.001***                              & -0.007***                             & 0.0003***                             & 0,0002                                & 0.0002***                             & -0.002***                             & 0,0002                                & -0,001                                & 0.002***                              & -0.003***                             & 0.0004***                             & -0,0003                               \\
\textbf{Factory orders}           & 0.002***                              & -0.013***                             & 0,0001                                & 0.001***                              & 0.0003***                             & -0.001***                             & 0,0001                                & -0.005***                             & 0.001***                              & -0.002*                               & 0,0001                                & 0.001*                                \\
\textbf{GDP}                      & 0.001**                               & -0.006**                              & 0.0003***                             & 0.001*                                & 0.0002***                             & 0,0002                                & -0.0003*                              & -0.002**                              & 0.001***                              & -0.002***                             & 0.001***                              & -0,00004                              \\
\textbf{Housing starts}           & -0,0001                               & 0,006                                 & -0.0005*                              & 0.002*                                & 0,0002                                & -0,0004                               & -0,001                                & -0,003                                & -0.014***                             & 0.028***                              & -0,0005                               & 0,004                                 \\
\textbf{Industrial production}    & 0.001***                              & -0.012***                             & 0.0002**                              & 0,0002                                & 0,0001                                & -0.001*                               & -0.0002*                              & -0,002                                & 0.001***                              & -0.003***                             & 0.0004***                             & -0,001                                \\
\textbf{Michigan Sentiment Index} & 0.001***                              & -0.006***                             & -0,00004                              & 0.001***                              & 0,00003                               & 0,0001                                & -0.0002*                              & 0,001                                 & 0.0001                                & 0.0002                                & 0,0001                                & 0,001                                 \\
\textbf{New home sales}           & 0.001**                               & -0.005*                               & 0.0004***                             & 0,0001                                & 0.0002***                             & 0,0001                                & -0,0001                               & -0,001                                & 0.001***                              & -0.002***                             & 0.001***                              & -0,0002                               \\
\textbf{Non-farm employment}      & 0.003***                              & -0.017***                             & 0.002***                              & 0,0003                                & 0.001***                              & -0.004***                             & 0.001***                              & -0.010***                             & 0.002***                              & -0.001                                & 0.004***                              & -0,001                                \\
\textbf{Pending home sales}       & 0.001***                              & -0.006**                              & 0.0002***                             & -0.001*                               & 0.0003***                             & 0,0003                                & -0,0001                               & -0.003***                             & 0.001**                               & -0.001                                & 0,0002                                & 0,001                                 \\
\textbf{Personal consumption}     & 0,00002                               & -0,0005                               & 0.0002*                               & -0,0003                               & -0,0001                               & 0,00002                               & 0,0001                                & -0.003***                             & 0.001**                               & -0.001                                & 0,00001                               & 0,001                                 \\
\textbf{Personal income}          & 0.001*                                & 0,002                                 & -0,0001                               & 0.001**                               & 0.0004***                             & 0.003***                              & -0,0002                               & 0.009***                              & -0.001                                & 0.002*                                & -0.0004*                              & 0,002                                 \\
\textbf{Producer price index}     & 0,001                                 & -0,003                                & 0.0002**                              & -0,0001                               & -0,0001                               & 0,0001                                & 0,0001                                & -0,0003                               & 0.001***                              & -0.001                                & 0.001***                              & -0,001                                \\
\textbf{Trade balance}            & 0,0004                                & -0,001                                & -0.001***                             & 0.003***                              & -0.0002***                            & 0,001                                 & 0,0001                                & 0,001                                 & 0.0004                                & -0.0004                               & -0.001***                             & 0.004***                              \\ \bottomrule
\end{tabular}%
}

\begin{tablenotes}
        \singlespacing
        \footnotesize
       Presents the estimate of eq. \ref{eq:Variance} using financialization variable NLS (for Money Manager positions). The $\Phi_m$ coefficients capture the instantaneous change in the conditional variance when an announcement has just occurred. The $\phi_m$ coefficients capture the conditional variance when an announcement has just occurred in conjunction with the level of financialization.  For all the regressions estimated in this table, we include the control variable NBER to control for the business cycle. 


    \end{tablenotes}
    
\end{table}
\end{landscape}

\begin{landscape}
\begin{table}[]
\caption{ Announcement and financialization effects on commodity futures conditional variance: Results based on a financialization variable built using swap dealer positions }
\label{tab:reg_vol_swap}
\resizebox{\columnwidth}{!}{%
\begin{tabular}{@{}lllllllllllll@{}}
\toprule
\textbf{Commodities}              & \multicolumn{2}{c}{\textbf{Crude Oil}}                                        & \multicolumn{2}{c}{\textbf{Gold}}                                             & \multicolumn{2}{c}{\textbf{Copper}}                                           & \multicolumn{2}{c}{\textbf{Natural Gas}}                                      & \multicolumn{2}{c}{\textbf{Palladium}}                                        & \multicolumn{2}{c}{\textbf{Silver}}                                           \\ \midrule
\textbf{Announcements}            & \multicolumn{1}{c}{\textbf{$\Phi_m$}} & \multicolumn{1}{c}{\textbf{$\phi_m$}} & \multicolumn{1}{c}{\textbf{$\Phi_m$}} & \multicolumn{1}{c}{\textbf{$\phi_m$}} & \multicolumn{1}{c}{\textbf{$\Phi_m$}} & \multicolumn{1}{c}{\textbf{$\phi_m$}} & \multicolumn{1}{c}{\textbf{$\Phi_m$}} & \multicolumn{1}{c}{\textbf{$\phi_m$}} & \multicolumn{1}{c}{\textbf{$\Phi_m$}} & \multicolumn{1}{c}{\textbf{$\phi_m$}} & \multicolumn{1}{c}{\textbf{$\Phi_m$}} & \multicolumn{1}{c}{\textbf{$\phi_m$}} \\ \midrule
\textbf{Initial jobless claims}   & 0.0004***                             & 0.002***                              & 0.0003***                             & -0.0005**                             & -0.0002*                              & 0.001***                              & -0.0002*                              & -0,001                                & 0.001***                              & 0.002***                              & 0.001***                              & 0.002***                              \\
\textbf{ADP Employment}           & 0,0001                                & 0,001                                 & 0.0004***                             & 0,0003                                & -0,0002                               & 0.002**                               & 0,0002                                & -0,001                                & -0.0002*                              & -0.002*                               & 0.0005***                             & -0,001                                \\
\textbf{CB Consumer}              & 0.001***                              & 0.002**                               & 0.0003***                             & 0,0004                                & -0.001***                             & 0.003***                              & 0,00002                               & -0,001                                & 0,0001                                & -0,0003                               & 0.0004***                             & 0.001*                                \\
\textbf{Advance retail sales}     & 0.001***                              & 0.006***                              & 0.001***                              & 0,0003                                & -0.001***                             & 0.005***                              & 0.001***                              & -0.004**                              & 0,0001                                & -0,001                                & 0.001***                              & -0,0005                               \\
\textbf{Building permit}          & -0,0001                               & 0,003                                 & 0,0004                                & 0,002                                 & 0                                     & 0,0001                                & 0,001                                 & -0,003                                & 0.004***                              & 0.041***                              & 0,0001                                & 0,002                                 \\
\textbf{Construction spending}    & 0.001***                              & 0.002**                               & 0.001***                              & -0,0003                               & -0.001***                             & 0.005***                              & 0,00003                               & 0,001                                 & 0.001***                              & 0.002*                                & 0.001***                              & 0.002**                               \\
\textbf{Consumer credit}          & 0,0002                                & 0.002*                                & 0,0001                                & 0,0002                                & 0,0002                                & -0,001                                & -0,0001                               & 0,0005                                & -0,0002                               & -0.002**                              & 0.0002**                              & -0,001                                \\
\textbf{Consumer price index}     & 0,00003                               & 0,001                                 & 0.001***                              & 0.002***                              & 0,0002                                & -0,001                                & -0,00003                              & 0,0001                                & 0,00001                               & -0,001                                & 0.001***                              & -0.002*                               \\
\textbf{Durable goods orders}     & 0.0005***                             & 0.002**                               & 0,0001                                & -0,0002                               & -0.0003*                              & 0.001*                                & -0,0001                               & 0,002                                 & 0.0004***                             & 0,001                                 & 0.0002**                              & 0.002**                               \\
\textbf{Existing home sales}      & 0.0003**                              & -0.002*                               & 0.0003***                             & -0,001                                & -0.0003*                              & 0.002***                              & 0,0003                                & -0,001                                & 0.001***                              & 0.004***                              & 0.0004***                             & 0,001                                 \\
\textbf{Factory orders}           & 0.001***                              & 0,001                                 & 0.0002***                             & -0.001***                             & -0.0004**                             & 0.002***                              & -0,0002                               & 0.005***                              & 0.0004***                             & 0.002**                               & 0.0002***                             & -0.002*                               \\
\textbf{GDP}                      & 0.0004***                             & 0.004***                              & 0,0001                                & -0.002***                             & -0,00001                              & 0,001                                 & -0,0003                               & 0,001                                 & 0.0005***                             & 0,002                                 & 0.001***                              & 0,001                                 \\
\textbf{Housing starts}           & 0,0004                                & 0,001                                 & -0,0003                               & -0,002                                & -0,001                                & 0,003                                 & -0,001                                & 0,004                                 & -0.003***                             & -0.040***                             & 0,00002                               & -0,001                                \\
\textbf{Industrial production}    & 0.0003**                              & 0.004***                              & 0.0004***                             & 0.001*                                & -0,0001                               & 0,0005                                & -0,0002                               & 0,0002                                & 0,0002                                & 0.002***                              & 0.0003***                             & 0,001                                 \\
\textbf{Michigan Sentiment Index} & 0,0001                                & 0,001                                 & 0,00002                               & -0.001***                             & 0,0001                                & -0,0003                               & 0,00001                               & -0,002                                & 0,0001                                & -0,001                                & 0.0003***                             & -0,0003                               \\
\textbf{New home sales}           & 0.0004***                             & 0.003***                              & 0.0004***                             & 0,0001                                & -0,00004                              & 0,001                                 & -0,0002                               & 0,002                                 & 0.0004***                             & 0.003***                              & 0.001***                              & 0.002**                               \\
\textbf{Non-farm employment}      & 0.002***                              & 0.013***                              & 0.003***                              & 0.002***                              & -0.002***                             & 0.011***                              & 0.001**                               & 0,003                                 & 0.002***                              & 0.004***                              & 0.003***                              & 0.007***                              \\
\textbf{Pending home sales}       & 0.0004***                             & 0.002*                                & 0.0001*                               & 0,0001                                & -0,0002                               & 0.002**                               & 0,0002                                & -0,001                                & 0.0003**                              & 0,001                                 & 0.0003***                             & -0,0003                               \\
\textbf{Personal consumption}     & -0,0001                               & -0,001                                & 0.0004***                             & 0.002***                              & 0,0001                                & -0,001                                & -0.0004*                              & 0.006***                              & 0.0004***                             & 0.002**                               & 0,0001                                & -0,001                                \\
\textbf{Personal income}          & 0.001**                               & -0.004**                              & -0.0005***                            & -0.004***                             & 0.001***                              & -0.003**                              & 0,0001                                & -0,004                                & 0,0001                                & -0,002                                & -0,00002                              & 0,0002                                \\
\textbf{Producer price index}     & 0,0001                                & -0,0003                               & 0,0001                                & -0.001**                              & -0,00002                              & -0,0003                               & 0,0001                                & -0,0001                               & 0.001***                              & 0.002**                               & 0.0004***                             & -0,001                                \\
\textbf{Trade balance}            & 0.0004***                             & 0,0002                                & -0.0004***                            & -0.004***                             & 0,0001                                & -0,001                                & 0,0001                                & -0,001                                & 0,0001                                & -0,001                                & 0,00002                               & -0.009***                             \\ \bottomrule
\end{tabular}%
}

\begin{tablenotes}
        \singlespacing
        \footnotesize
       Presents the estimate of eq. \ref{eq:Variance} using financialization variable NLS (for swap dealers positions). The $\Phi_m$ coefficients capture the instantaneous change in the conditional variance when an announcement has just occurred. The $\phi_m$ coefficients capture the conditional variance when an announcement has just occurred in conjunction with the level of financialization.For all the regressions estimated in this table, we include the control variable NBER to control for the business cycle. 


    \end{tablenotes}
    
\end{table}
\end{landscape}










\end{document}
